\documentclass[../notes.tex]{subfiles}
\begin{document}
\section{Day 5}
    \subsubsection{To what extent does $\pi_1$ depend on $x$?}
    \begin{theorem} Let $X$ be a path-connected topological space, and let $x_0,x_1\in X$, then
    $\pi_1(X,x_0)\cong\pi_1(X,x_1)$. This section builds off the worksheet provided in class.
    \begin{enumerate}
        \item \underline{Part 1}: see drawing
        \item \underline{Part 2}: Let $f$ and $g$ be in $\pi_1(X,x_1)$
            \begin{align*}
                \hat{\alpha}([f]*[g])
                &=[\overline{\alpha}] *[f]*[g] *[\alpha]\\
                &=[\overline{\alpha}] *[f] *[\alpha] *[\overline{\alpha}] *[g] *[\alpha]\\
                &=\hat{\alpha}([f])*\hat{\alpha}([g])\\
            \end{align*}
        \item \underline{Part 3}: Let $f \in \pi_1(X,x_1)$
            \begin{align*}
                \hat{\alpha}([f])
                &=[\overline{\alpha}] *[f] *[\alpha]\\
                \hat{\overline{\alpha}}( [\overline{\alpha}] *[f] *[\alpha])
                &=[\alpha] *[\overline{\alpha}] *[f] *[\alpha] *[\overline{\alpha}]\\
                &=[f]\\
            \end{align*}
        \item Therefore this mfer is an isomorphism.
    \end{enumerate}
    \end{theorem}
    \subsubsection{ For which topological spaces $X$ can we actually compute $\pi_1(X,x)$?}
    \begin{definition} A topological space $X$ is \underline{simply-connected} if
    \begin{enumerate}
        \item $X$ is path connected
        \item $\pi_1(X,x)={1}\ \forall x\in X$\\
            (Because $X$ is path connected, we only need to check this for one $x\in X$)
    \end{enumerate}
    \end{definition}
    \begin{itemize}
        \item \underline{Ex:} $\R^2$ is simply connected
        \item \underline{Intuition:} $X$ is simply-connected if any loop in $X$ if any loop in $X$ can
            be ``shrunk down'' to a constant loop.\\
            (for all loops $f$ in $X$ saying $f$ can be ``shrunk down'' means $f\cong_{p}c_x$ where $c_x$ is a constant path)
        \item \underline{Next time:} A convex subset of $\R^n$ is simply connected.
    \end{itemize}
\end{document}