\documentclass[../notes.tex]{subfiles}
\begin{document}
\section{Day 27}
Recall from yesterday,
\begin{enumerate}
    \item Simplicial chain complex from worksheet
        \[
            \{0\}\rightarrow C_2(k)
            \rightarrow^{\partial_2}C_1(k)
            \rightarrow^{\partial_1}C_0(k)
            \rightarrow \{0\}
        \]
    \item 
        \begin{align*}
            \partial_1(e_1) = \partial_1([v_1,v_2]) = v_2-v_1\\
            \partial_1(e_2) = \partial_1([v_1,v_3]) = v_3-v_1\\
            \partial_1(e_3) = \partial_1([v_2,v_3]) = v_3-v_2\\
            \partial_2(f_1) = e_3 - e_2 + e_1\\
        \end{align*}
    \item 
        \begin{align*}
            \partial_2([v_1,v_2,v_3])=[v_2,v_3]-[v_1,v_3]+[v_1,v_2]\\
            \partial_2([v_1,v_2,v_3])=e_3-e_1+e_2
        \end{align*}\\
        \begin{center}
            \begin{tikzpicture}
                \draw[->] (0,0) node [below left] {$p_1$} --
                          (0,6) node [above]{$p_2$} --
                          (6,0) node [below right] {$p_3$} -- cycle;
                          \draw (3,0) node [below] {$e_1$}\\
                          \draw (0,3) node [left] {$e_2$}\\
                          \draw (3,3) node [above right] {$e_3$}\\
                          \draw (1.5,1.5) node [above right] {$f_1$}\\
            \end{tikzpicture}
        \end{center}
    \item 
        \begin{align*}
            ker(\partial_2)=\{af_1| ker(af_1)=0\}
        \end{align*}
        Since $\partial_2(f_1)\neq 0$, the only multiple of $f_1$ sent to 0 is $0f_1$
    \item
        Kernels,
        \begin{align*}
            ker(\partial_1)&=\{ae_1-ae_2+ae_3|\ a\in \Z\}\\
            &=\inpr{e_1-e_2+e_3}\\
            ker(\partial_2)&=\{0\}
        \end{align*}
        Images,
        \begin{align*}
            im(\partial_2)=\inpr{e_3-e_2+e_1}\\
            im(\partial_1)=\inpr{v_2-v_1,v_3-v_1,v_3-v_2}\\
        \end{align*}
        Homologies,
        \begin{align*}
            H_2(K)=\frac{ker(\partial_2)}{im(\partial_3)}=\frac{\{0\}}{\{0\}}\\
            H_1(K)=\frac{ker(\partial_1)}{im(\partial_2)}=
            \frac{\inpr{e_1-e_2+e_3}}{\inpr{e_3-e_2+e_1}}\\
            H_0(K)=\frac{ker(\partial_0)}{im(\partial_1)}=
            \frac{C_0(K)}{\inpr{v_2-v_1,v_3-v_1,v_3-v_2}}\cong \Z\\
        \end{align*}
        Where,
        \[
            a_1v_1+a_2v_2+a_3v_3\in C_0(K)
        \]
        Note that this is like setting,
        \begin{align*}
            v_2-v_1=0\\
            v_3-v_1=0\\
            v_3-v_2=0\\
        \end{align*}
    \item \underline{Ex:} Given the following simplicial complex,
        \begin{center}
            \begin{tikzpicture}
                \draw[->] (0,0) node [below left] {$p_1$} --
                          (0,6) node [above]{$p_2$} --
                          (6,0) node [below right] {$p_3$} -- cycle;
                          \draw (3,0) node [below] {$e_1$}\\
                          \draw (0,3) node [left] {$e_2$}\\
                          \draw (3,3) node [above right] {$e_3$}\\
            \end{tikzpicture}
        \end{center}
        \[
            \{0\}
            \rightarrow^{\partial_2}C_1(K)
            \rightarrow^{\partial_1}C_0(K)
            \rightarrow \{0\}
        \]
        With,
        \begin{align*}
            C_1(K)\cong \Z^3\\
            C_0(K)\cong \Z^3
        \end{align*}
        $ker(\partial_1)$ and $im(\partial_2)$ are as before, but now,
        \begin{align*}
            ker(\partial_2)=\{0\}\\
            im(\partial_2)=\{0\}
        \end{align*}
        Meaning that our homology ends up as,
        \begin{align*}
            H_2(K)&=\frac{ker(\partial_2)}{im(\partial_3)}=\frac{\{0\}}{\{0\}}\\
            H_1(K)&=\frac{ker(\partial_1)}{im(\partial_2)}=\frac{\inpr{e_1-e_2+e_3}}{\{0\}}
            \cong \Z\\
            H_0(K)&=\frac{ker(\partial_0)}{im(\partial_1)}=
            \frac{C_0(K)}{\inpr{v_2-v_1,v_3-v_1,v_3-v_2}}\cong \Z\\
        \end{align*}
    \item \underline{Note:} Just like $\pi_1$ $H_1$ detected the ``holes'' in the 2nd
        example. In general, nonzero elements of $H_n(X)$:
        \begin{align*}
            ker(\partial_n)&\ni \text{$n$-chains that \underline{could be the boundary
                    of an $(n+1)$-simplex}}\\
            \text{But aren't } &\notin im(\partial_{n+1})
        \end{align*}
    \item \underline{Ex:}
        \begin{center}
            \begin{tikzpicture}
                \draw[->] (0,0) node --
                          (0,6) node --
                          (6,0) node -- cycle;
                \draw[->] (6,6) node --
                          (0,6) node --
                          (6,0) node -- cycle;
            \end{tikzpicture}
        \end{center}
        $H_1(K)\cong \Z^2$
\end{enumerate}
\begin{itemize}
    \item Option 1: Given $X$ look for a simplicial complex $K$ such that $|K|\cong X$
        and define,
        \[
            H_n(X)=H_n(K)
        \]
    \item Option 2: ``triangulate'' X
\end{itemize}
\begin{definition}
    The \underline{standard n-simplex}
    \[
        \Delta^n= \text{ simplex spanned by, $
            \underbrace{(0,0,\dots,0)}_{v_0}
            \underbrace{(1,0,\dots,0)}_{v_1},\dots,
            \underbrace{(0,0,\dots,1)}_{v_n}$ in $\R^n$}
    \]
\end{definition}
\begin{itemize}
    \item \underline{Ex:} OH GREAT TIME TO DRAW SOME SIMPLICES REALLY GREAT JUST WONDERFUL
\end{itemize}
\begin{definition}
    The \underline{interior} of $\Delta^n$ is,
    \[
        int(\Delta^n)=\Delta^n\setminus \bigcup \{\text{ proper faces }\}
    \]
\end{definition}
\begin{definition}
    Let $x$ be any topological space. A \underline{$\Delta$-complex structure} on $X$
    is a collection of sets,
    \[
        S_n = \{\sigma_1^n,\dots,\sigma_{k_n}^n\}
    \]
    (one of these sets for each $n\in \Z^{\geq 0}$) where,
    \begin{align*}
        \sigma_i^n:\Delta^n\rightarrow X\\
    \end{align*}
    are continuous maps $\forall i$
    \begin{itemize}
        \item 
            \begin{align*}
                {\sigma_i^n}_{|int(\Delta^n)}
            \end{align*}
            is injective $\forall i,n$ and each $x\in X$ is in the image of exactly one
            ${\sigma_i^n}_{|int(\Delta^n)}$
    \end{itemize}
\end{definition}
\end{document}

