\documentclass[../notes.tex]{subfiles}
\begin{document}
\section{Day 31}
\subsection{Last time}
\begin{itemize}
    \item Proving,
        \begin{align*}
            \phi: H_0(X)\rightarrow \Z\\
            \phi([\alpha])=\text{ sum of coefficients in $\alpha$ }\\
        \end{align*}
        is an isomorphism when $X$ is path-connected
    \item \underline{Injectivity:} Suppose
        \begin{align*}
            \phi([\alpha])=0\\
            \Rightarrow \text{Sum of coefficients in $\alpha$}=0\\
            \Rightarrow [\alpha]=[\sigma_1-\sigma_2, \sigma_3-\sigma_4,\dots,
                                  \sigma_{r-1}-\sigma_r]\\
        \end{align*}
        E.g. if
        \begin{align*}
            \alpha=3\sigma_1-2\sigma_2-\sigma_3\\
            =\sigma_1+\sigma_1+\sigma_1-\sigma_2-\sigma_2-\sigma_3\\
            =\sigma_1-\sigma_2+\sigma_1-\sigma_2+\sigma_1-\sigma_3\\
        \end{align*}
        Each $\sigma_i:\Delta^0\rightarrow X$ is a constant map from $\Delta^0$ to
        a single point, $x_i\in X$. Choose a path $\delta_i:I\rightarrow X$ from
        $x_{i+1}$ to $x_i$. This is a 1-simplex in $X$, namely,
        \begin{align*}
            \tau_i:\Delta^1\rightarrow X\\
            \tau_i(1-t,t)=\delta_i(t)\\
        \end{align*}
        Then,
        \begin{align*}
            \partial_1(\tau_i)=\tau_i\circ f_0-\tau_i\circ f_1\\
            =\underbrace{{\tau_i}_{|t=1}}_{\text{constant map at $x_i$}}-
            \underbrace{{\tau_i}_{|t=1}}_{\text{constant map at $x_{i+1}$}}\\
            =\sigma_i-\sigma_{i+1}
        \end{align*}
        Thus,
        \begin{align*}
            [\alpha]=[
            \underbrace{\sigma_1-\sigma_2}_{\partial_1(\tau_1}+
            \underbrace{\sigma_3-\sigma_4}_{\partial_1(\tau_2}+
            \dots+
            \underbrace{\sigma_{r-1}-\sigma_r}_{\partial_1(\tau_{t-1}})]\\
            =[\partial_1(\tau_1+\dots+\tau_{t-1})]\\
        \end{align*}
    \item \underline{Surjectivity:} For any $a\in \Z$,
        \[
            \phi([a\cdot\text{any $\sigma$-simplex}])=a
        \]
    \item \underbrace{Pictorially:} 
        \begin{center}
            \LARGE{SOME GODDAMN PICTURE I CAN'T MAKE YET AHH FUCK}
        \end{center}
        \begin{align*}
            H_0(X)&=\frac{C_0(X)}{im(\partial_1)}\\
            \partial_1(\tau)&=w-v\\
            &\implies w-v\in im(\partial_1)\\
            &\implies [w]=[v]\in H_0(X)\\
            &\implies \text{every generator of $C_0$ equals every other}
        \end{align*}
\end{itemize}
\begin{theorem}
    Suppose,
    \[
        X=X_1\sqcup X_2 \sqcup \dots \sqcup X_k\\
    \]
    Where $X_1,\dots,X_k$ are path-connected (``path components of X''). Then,
    \[
        H_n(X)\cong H_n(X_1)\oplus H_n(X_2)\oplus\dots\oplus H_n(X_k),\ \forall n\geq 0
    \]
\end{theorem}
\begin{corollary}
    \[
        H_0(X)\cong \Z^{\text{$\#$ number of path components}}
    \]
\end{corollary}
\begin{proof}
    For simplicity, let,
    \[
        X=X_1\sqcup X_2
    \]
    First, claim
    \[
        C_n(X)\cong C_n(X_1)\oplus C_n(X_2)
    \]
    To see this, let:
    \begin{align*}
        \phi: C_n(X)\cong C_n(X_1)\oplus C_n(X_2)\rightarrow C_n(X)\\
        \phi(\alpha_1,\alpha_2)=(i_1)_{\#}\alpha_1+(i_2)_{\#}\alpha_2\\
    \end{align*}
    Where $i_1:X_1\rightarrow X$ and $i_2:X_2\rightarrow X$ are the inclusions.
    Note that $\alpha_1$ and $\alpha_2$ are linear combinations of simplices in their 
    respective components.\\
    This is
    \begin{itemize}
        \item Injective, because the images of $(i_1)_{\#}(\alpha_1)$ and
            $(i_2)_{\#}(\alpha_2)$ are disjoint, so no cancellation can occur.
        \item Surjective, because if,
            \[
                \alpha=a_1\sigma_1+\dots+\a_k\alpha_r\in C_n(X)
            \]
            $\implies$ each $\sigma_i$ must land entirely in either $X_1$ or $X_2$,
            (Because $\sigma_i:\Delta^n\rightarrow X$ is continuous and $\Delta^n$ is
            path-connected)\\
            \begin{align*}
                \implies& \alpha=(i_1)_{\#}(\text{stuff in $X_1$})+(i_2)_{\#}(\text{stuff in $X_2$})\\
                \implies& \alpha\in im(\phi)\\
            \end{align*}
    \end{itemize}
    Thus, $C_n(X)\cong C_n(X_1)\oplus C_n(X_2)$. And $\partial_n$ respects this
    decomposition:\\
    \begin{center}
        \begin{tikzcd}
            C_n(X)\arrow[r, "\partial_n"]& C_{n-1}(X)\\
            C_n(X_1)\arrow[r,"\partial_n^1"]&C_{n-1}(X_1)\\
            C_n(X_2)\arrow[r,"\partial_n^2"]&C_{n-1}(X_2)\\
        \end{tikzcd}
    \end{center}
    \begin{align*}
        C_n(X)\cong C_n(X_1)\oplus C_n(X_2)\\
        C_{n-1}(X)\cong C_{n-1}(X_1)\oplus C_{n-1}(X_2)
    \end{align*}
    Thus,
    \begin{align*}
        H_n(X)=\frac{ker(\partial_n)}{im(\partial_{n+1})}
        =\frac{ker(\partial_n^1)\oplus ker(\partial_n^2)}{
            im(\partial_{n+1}^1)\oplus im(\partial_{n+1}^2)
        }
    \end{align*}
\end{proof}
\begin{itemize}
    \item The last homology we can compute by force is:
\end{itemize}
\begin{theorem}
    Let $X=\{\cdot\}$. Then,
    \[
        H_n(X)=\{0\}
    \]
    for all $n\gt 0$
\end{theorem}
\begin{proof}
    For any $n$ there's only one possible $n$-simplec,
    \begin{align*}
        \sigma^n:\Delta^n\rightarrow X\\
        \sigma^n=\text{ constant map }
    \end{align*}
    Thus,
    \[
        C_n(X)=\{a\cdot \sigma^n|\ a\in \Z\}\cong \Z
    \]
    The boundary maps are,
    \begin{align*}
        \partial_n: C_n(X)\rightarrow C_{n-1}\\
        \partial_n(\sigma^n)=\sum_{j=0}^n(-1)^j)(\underbrace{\sigma^n\circ f_j}_{
            (n-1)\text{-simplices, so must $=\sigma^{n-1}$}
        })\\
        =\sigma^{n-1} -\sigma^{n-1} +\sigma^{n-1} -\sigma^{n-1} \dots +(-1)^n\sigma^{n-1}\\
    \end{align*}
\end{proof}
\end{document}
