\documentclass[../notes.tex]{subfiles}
\begin{document}
\section{Day 22}
Recall the Van Kampen theorem: (on page 38 of these notes)
\begin{itemize}
    \item \underline{Ex:}
        \begin{align*}
            X = \text{torus} = \text{Complicated scribble} =
            \frac{[0,1]\times [0,1]}{\sim}\\
            (0,y)\sim(1,y)\\
            (x,1)\sim(x,1)\\
        \end{align*}
        \begin{align*}
            U = \{[p]\in \frac{[0,1]\times [0,1]}{\sim}|\ p\in B_{\frac{1}{3}}(
                \frac{1}{2},\frac{1}{2})\}\\
            V = \{[p]\in \frac{[0,1]\times [0,1]}{\sim}|\ p\notin B_{\frac{1}{4}}(
                \frac{1}{2},\frac{1}{2})\}\\
        \end{align*}
        Then, with complicated doodles afoot,
        \begin{align*}
            U\simeq \{.\}\Rightarrow \pi_1(U,x_0)=\{1\}\\
            V\simeq \{\text{a drawing}\}\Rightarrow \pi_1(V,x_0)=\inpr{a,b}\\
            U\cap V \simeq S^1\Rightarrow \pi_1(U\cap V, x_0)=\Z\\
        \end{align*}
        The homomorphisms induced by $i_U$ and $i_V$ are,
        \begin{align*}
            i_{U*}:\underbrace{\pi_1(U\cap V, x_0)}_{=\Z}\rightarrow
            \underbrace{\pi_1(U,x_0)}_{=\{1\}}
            i_{U*}(g)=1,\ \forall g\\
        \end{align*}
        With the other as,
        \begin{align*}
            i_{V*}:\underbrace{\pi_1(U\cap V, x_0)}_{=\Z=\inpr{c}}
                \rightarrow \underbrace{\pi_1(V,x_0)}_{\inpr{a,b}}\\
        \end{align*}
        There is some inscrutable drawing here that encapsulates the logic,
        maybe I'll add it. Who knows? I sure don't!
        \begin{align*}
            i_{V*}(c)=aba^{-1}b^{-1}\\
        \end{align*}
        So, the Van Kampen Theorem says,
        \begin{align*}
            \pi_1(X)\cong \{1\}*_{\inpr{c}}\inpr{a,b}\\
            =\frac{\{1\}*\inpr{a,b}}{\text{smallest yadda yadda yadda}}\\
        \end{align*}
        With the smallest yadda yadda yadda containing,
        \[
            i_{U*}(h)i_{V*}(h),\ \forall h\in \inpr{c}
        \]
        In this quotient,
        \begin{align*}
            aba^{-1}b^{-1}=1\\
            \iff ab=ba\\
        \end{align*}
        Thus, in this quotient, any word is equivalent to a word of the form,
        \[
            a^kb^l
        \]
        Where $k,l\in \Z$, which is to say that,
        \[
            \pi_1(X,x_0)\cong \Z \oplus \Z
        \]
    \item \underline{Question:} What is the $\pi_1(K)$ where $K$ is a klein bottle?\\
        (unanswered at the moment)
\end{itemize}
\end{document}
