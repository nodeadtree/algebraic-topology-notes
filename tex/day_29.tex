\documentclass[../notes.tex]{subfiles}
\begin{document}
\section{Day 29}
\begin{itemize}
    \item \underline{So far:} Defined $H_n^{\Delta}(X)$, the simplicial homology of a space
        $X$ with a $\Delta$-complex structure
    \item \underline{Issues:}
        \begin{itemize}
            \item Does it depend on the $\Delta$-complex structure?
            \item $X\cong_{homeo}Y\implies H_n^{\Delta}(X)\cong H_n^{\Delta}(Y)$?
            \item Does $f:X\rightarrow Y$ induce $f_*:H_n^{\Delta}(X)\rightarrow H_n^{\Delta}(Y)$?
        \end{itemize}
        To resolve these issues, we'll replace simplicial homology with a (seemingly much larger), version
        of homology.
\end{itemize}
\subsection{Singular Homology}
Let the following be so,
\begin{align*}
    X=\text{ any topological space }\\
    \Delta^n=\text{ standard $n$-simplex }\subseteq \R^{n+1}\\
\end{align*}
\begin{definition}
    \begin{align*}
        S_n(X)=\{\text{continuous maps from $\sigma:\Delta^n\rightarrow X$}\}\\
    \end{align*}
    Note that, $S_n(X)$ is absurdly huge and that our continuous maps can be 
    singularly horrible.
    \[
        C_n(X)=\{a_1\sigma_1+\dots+a_r\sigma_r|\ a_1,\dots,a_r\in \Z,\ \sigma_1,\dots,\sigma_r\in S_n(X)\}
    \]
    Note that this is equivalent to a product of uncountably many copies of $\Z$
    We call $C_n(X)$ the group of \underline{singular $n$-chains}
\end{definition}
\begin{definition}
    The (singular) \underline{boundary homomorphisms} are:
    \begin{align*}
        \partial_n:C_n(X)\rightarrow C_{n-1}(X)\\
        \partial_n(\sigma)=\sum_{j=0}^n(-1)^{j}(\sigma\circ  f_j)\\
    \end{align*}
\end{definition}
\begin{definition}
    The \underline{singular homology} of $X$ is,
    \begin{align*}
        \frac{ker(\partial_n)}{im(\partial_{n+1})}
    \end{align*}
\end{definition}
\begin{itemize}
    \item \underline{Note:} The ``same'' proof for simplicial homology shows,
        \[
            \partial_n\circ\partial_{n+1}=0
        \]
        This $\implies$
        \[
            im(\partial_n+1)\subseteq ker(\partial_n)
        \]
        So, the definition of $H_n(x)$ makes sense.
    \item So, what's craaaazy about this definition,
        \[
            H_n(X)=\frac{\Z^{\text{uncountable}}}{\Z^{\text{uncountable}}}
        \]
        but crazily:
\end{itemize}
\begin{theorem}
    \[
        H_n(X)\cong H_n^{\Delta}(X)
    \]
    for any $\Delta$-complex structure on $X$
\end{theorem}
\begin{itemize}
    \item We won't prove this theorem, but we'll see some examples.
    \item \underline{Note:} This theorem implies that $H_n^{\Delta}(X)$
        is independent of the $\Delta$-complex structure you choose. Look
        into relative homology to show some of this I guess? Read chapter 2 of Hatcher
        "The equivalence of simplicial and singular homology"
    \item \underline{Note:} This theorem implies that $H_n^{\Delta} (X)$ is
        independent of the $\Delta$-complex structure you choose.
\end{itemize}
\subsection{Induced Homomorphisms}
\begin{itemize}
    \item
        Let $f:X\rightarrow Y$ be continuous
    \item Then $f$ induces a homomorphism,
        \begin{align*}
            f_{\#}&:C_n(X)\rightarrow C_n(Y)\\
            a_1\sigma_1&+\dots a_r\sigma_r\in C_n(X),\ \text{etc.}\\
            f_{\#}(a_1\sigma_1+\dots+a_r\sigma_r)&=
            a_1(f\circ \sigma_1)+\dots+a_r(f\circ \sigma_r)\\
        \end{align*}
        With $\sigma_i$ and $f\circ \sigma_i$ are maps,
        \begin{align*}
            \sigma_i:\Delta^n\rightarrow X\\
            f\circ\sigma_i:\Delta^n\rightarrow Y\\
        \end{align*}
        To check that this induces a homomorphism,
        \[
            f_*:H_n(X)\rightarrow H_n(Y)
        \]
        With,
        \begin{align*}
            H_n(X)=\frac{ker(\partial_n^X)}{im(\partial_{n+1}^X)}\\
            H_n(Y)=\frac{ker(\partial_n^Y)}{im(\partial_{n+1}^Y)}\\
        \end{align*}
        we'll need to check the fact that $f_{\#}$ takes $ker(\partial_n^X)$ to
        $\partial_n^Y$ and $im(\partial_{n+1}^X)$ to $im(\partial_{n+1}^Y)$
    \item These both follow from:
\end{itemize}
\begin{lemma}
    Let,
    \begin{align*}
        \partial_n^X:C_n(X)\rightarrow C_{n-1}(X)\\
        \partial_n^Y:C_n(Y)\rightarrow C_{n-1}(Y)\\
    \end{align*}
    be the boundary homomorphisms, then,
    \[
        \partial_n^Y\circ f_{\#}=f_{\#}\circ\partial_n^X
    \]
\end{lemma}
\begin{proof}
    Suffices to prove on $n$-simplices $\sigma$:
    \begin{align*}
        (\partial_n^Y\circ f_{\#})(\sigma)&=\partial_n^Y(f\circ \sigma)\\
        &=\sum_{j=0}^n(-i)^j(f\circ \sigma)\circ f_j\\
        &=\sum_{j=0}^n(-i)^jf\circ(\sigma\circ f_j)\\
        &=f_{\#}(\sum_{j=0}^n(-i)^j(\sigma\circ f_j))\\
        &=f_{\#}(\partial_n^X(\sigma))
    \end{align*}
\end{proof}
\begin{itemize}
    \item We'll show next time that,
        \begin{align*}
            f_*&:H_n(X)\rightarrow H_n(Y)\\
            f_*([a_1\sigma_1+\dots+a_r\sigma_r])&=[f_{\#}(a_1\sigma_1+\dots+a_r\sigma_r)]\\
        \end{align*}
\end{itemize}
\end{document}
