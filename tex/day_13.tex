\documentclass[../notes.tex]{subfiles}
\begin{document}
\section{Day 13}
    \begin{theorem} 
        Let
        \begin{align*}
            f: X\rightarrow Y\\
            g: X\rightarrow Y\\
        \end{align*}
        be a continuous function such that $f(x_0)=g(x_0)=y_0$. Suppose that
        $f$ and $g$ are \underline{homotopic relative to $x_0$}.\\(there exists a homotopy
        $H$ from $f$ to $g$ such that $H(x_0, t)=y_0,\ \forall t$).\\
        Then,
        \begin{align*}
            f_*:\pi_1(X,x_0)\rightarrow \pi_1(Y,y_0)\\
            f_*:\pi_1(X,x_0)\rightarrow \pi_1(Y,y_0)\\
        \end{align*}
        are the same homomorphism.
    \end{theorem} 
    \begin{proof}
        Let $[\alpha]\in \pi_1(X,x_0)$. We want,
        \begin{align*}
            f*[\alpha]=g*[\alpha]\\
            \iff [f\circ \alpha]=[g\circ \alpha]\\
            \iff f\circ \alpha \cong_{p} g\circ\alpha\\
        \end{align*}
        Define,
        \begin{align*}
            P: I\times I \rightarrow Y\\
            P(s,t)=H(\alpha(s), t)\\
        \end{align*}
        Equivalently, 
        \begin{align*}
            p_t&: I \rightarrow Y\\
            p_t(s)&=(h_t\circ \alpha)(s)\\
        \end{align*}
        This is a path homotopy from $f\circ \alpha$ to $g\circ\alpha$.
        Firstly, because $H$ is a homotopy relative to $x_0$.
        \begin{align*}
            P(0,t)=H(\alpha(0),t)=H(x_0,t)=y_0\\
            P(1,t)=H(\alpha(1),t)=H(x_0,t)=y_0\\
        \end{align*}
        Because $H$ is a homotopy from $f$ to $g$, the following is true.
        \begin{align*}
            P(s,0)=H(\alpha(s),0)=f(\alpha(s))\\
            P(s,1)=H(\alpha(s),1)=g(\alpha(s))\\
        \end{align*}
    \end{proof}
\begin{itemize}
    \item
        \underline{Application:} Suppose $A\subseteq X$ and that there exists
        a homotopy $H$ from
        \begin{align*}
            id:X\rightarrow X\\
        \end{align*}
        to a continuous function
        \begin{align*}
            r: X\rightarrow X\\
        \end{align*}
        such that,
        \begin{enumerate}
            \item $r(x)\in A,\ \forall x\in X$
            \item $H(a,t)=a,\ \forall a \in A,\ \forall t\in I$\\
                (``every point of $A$ stays fixed throughout the homotopy, or,
                $H$ is a homotopy relative to every point in $A$)
        \end{enumerate}
        In this situation, we say that $A$ is a \underline{deformation retract} of $X$
        or that $H$ is a \underline{deformation retraction} of $X$ onto $A$.
    \end{itemize}
    \begin{theorem} If $A$ is a deformation retract of $X$, then,
        \begin{align*}
            \pi_1(X,x_0)\cong \pi_1(A,x_0),\ \forall x_0\in A\\
        \end{align*}
    \end{theorem}
    \begin{itemize}
    \item\underline{Ex:}
        \begin{align*}
            X&=\R^2\\
            A&=S^1\\
            r&:X\rightarrow X\\
            r(x,y)&=(\frac{x}{\sqrt{x^2+y^2}},\frac{y}{\sqrt{x^2+y^2}})
        \end{align*}
        On Friday, we saw that the straight line homotopy, $H:X\times I\rightarrow X$
        is a homotopy from $id:X\rightarrow X$ to $r:X\rightarrow X$.
    \item \underline{Ex:}
        \begin{align*}
            X&=\{(x,y)\in \R^2| x^2+y^2\leq 1\}\\
            A&=\{(0,0)\}\\
            r&: X\rightarrow X\\
            r(x,y)&=(0,0)\\
        \end{align*}
        On Friday, we saw that the straight line homotopy $H:X\times I \rightarrow X$ is
        a homotopy from $id:X\rightarrow X$ to $r:X\rightarrow X$,
        Thus,
        \begin{align*}
            \pi_1(X)&\cong_{p}\pi_1(\{.\})=\{1\}\\
        \end{align*}
    \item \underline{Question:} Let,
        \begin{align*}
            X&=\R^3\setminus \{\text{z-axis}\}\\
            A&=\{(x,y,0)| x\neq 0,\ y\neq 0\}\\
        \end{align*}
        Find a deformation retraction from $X$ onto $A$. (Specify both $r$ and $H$)\\
        What does this tell us about $\pi_1(\R^3\{\text{z-axis}\})$
    \item \underline{Answer:}
        \begin{align*}
            r(x,y,z)&=(x,y,0)\\
            H((x,y,z),t)&=(x,y, (1-t)z)\\
        \end{align*}
        Thus,
        \begin{align*}
            \pi_1(\R^3\setminus\{\text{z-axis}\})\cong \pi_1(A)\cong \pi_1(\R^2\setminus\{(0,0)\})\cong \pi_1(S^1)\cong \Z\\
        \end{align*}
    \end{itemize}
    \begin{proof} Let $x_0\in A$. Let,
            \begin{align*}
                i&: A\rightarrow X\\
                i(a)&=a\\
                s&: X\rightarrow A\\
                s(x)&=r(x)\\
            \end{align*}
            Considering condition 2 in the definition
            of deformation retraction yields, $s\circ i= id_A$, because
            \begin{align*}
                s(i(a))=s(a)=a\\
            \end{align*}
            In the other direction,
            \begin{align*}
                i\circ s = r\\
            \end{align*}
            The deformation retraction $H$ is a homotopy relative to $x_0$ from $r$ to $id_X$, so:
            \begin{align*}
                r_*&=(id_X)_*\\
                \implies (i\circ s)_*&=(id_X)_*\\
                \implies i_* \circ s_*&=id\\
            \end{align*}
    \end{proof}
\end{document}
