\section{Day 12}
    \subsubsection{Homotopy of maps}
            \begin{definition}
            Let $f: X\rightarrow Y$ be a continuous function.
            A \underline{homotopy} from $f$ to $g$ is a continuous function,
            \begin{align*}
                H&:X\times I \rightarrow Y\\
                \text{such that}&\\
                H(x,0)&=f\\
                H(x,0)&=g\\
            \end{align*}
            We'll often write,
            \begin{align*}
                h_t&:X \rightarrow Y\\
                h_t(x)&=H(x,t)\\
            \end{align*}
            Then there's one $h_t$ for each $t\in I$ and,
            \begin{align*}
                h_0&=f\\
                h_1&=g\\
                h_t&=\text{``A function interpolating between $f$ and $g$''}\\
            \end{align*}
            \end{definition}
    \begin{itemize}
        \item \underline{Terminology/Notation:} If there exists a homotopy from
            $f$ to $g$, we'll say that \underline{$f$ is homotopic to $g$} and write
            $f\cong g$.\\
        \item \underline{Ex:}
            \begin{align*}
                f&: S^{1}\rightarrow \R^2\\
                g&: S^{1}\rightarrow \R^2\\
                f(x,y)&=(x,y)\\
                g(x,y)&=(0,0)\\
            \end{align*}
            Then $f\cong g$. A homotopy from $f$ to $g$ is,
            \begin{align*}
                H&: S^{1}\times I \rightarrow \R^2\\
                H((x,y),t)&=((1-t)x,(1-t)y)\\
            \end{align*}
            Do the drawing from the board.
        \item \underline{Ex:}
            \begin{align*}
                f&:\R \rightarrow \R\\
                g&:\R \rightarrow \R\\
                f(x)&=x\\
                g(x)&=x+2\\
            \end{align*}
            Then $f\cong g$. A homotopy from $f$ to $g$ is:
            \begin{align*}
                H&:\R \times I \rightarrow \R\\
                H(x,t)&=x+2t\\
            \end{align*}
            Refer again to the picture from the board.
            \newpage
        \item \underline{Questions:}
            \begin{itemize}
                \item
                    \begin{align*}
                        f&: \R \rightarrow \R^2\\
                        g&: \R \rightarrow \R^2\\
                        f(x)&=(x,0)\\
                        g(x)&=(x,e^x)\\
                    \end{align*}
                \item
                    \begin{align*}
                        f&: \R^2\setminus{(0,0)} \rightarrow \R^2\setminus{(0,0)}\\
                        g&: \R^2\setminus{(0,0)} \rightarrow \R^2\setminus{(0,0)}\\
                        f(x)&=(x,y)\\
                        g(x)&=(\frac{x}{\sqrt{x^2+y^2}},\frac{y}{\sqrt{x^2+y^2}})\\
                    \end{align*}
                    \begin{align*}
                        f&: \R \rightarrow \R^2\\
                        g&: \R \rightarrow \R^2\\
                        f(x)&=(x,0)\\
                        g(x)&=(x,e^x)\\
                    \end{align*}
                    Just use the straight line homotopy it's not hard.
            \end{itemize}
            Maybe include the drawings?
    \end{itemize}
        \begin{definition}
            Let $f:X\rightarrow Y$ and $g: X\rightarrow Y$ be continuous,
            and let $x_0 \in X$ be such that $f(x_0)=g(x_0)=y_0$. Then a
            \underline{homotopy from $f$ to $g$ relative to $x_0$} is a homotopy $H:X\times I \rightarrow Y$
            from $f$ to $g$ such that $h_t(x_0)=y_0,\ \forall t$.\\
            (``$x_0$ doesn't move during the homotopy'')
        \end{definition}
    \begin{itemize}
        \item \underline{Ex:} in the second part of the questions from today, $H$ was a homotopy relative to $(1,0)$, or
            to any other point on the unit circle.
        \item \underline{Ex:}
            \begin{align*}
                X=\{(x,y)\in \R^2 | x^2+y^2\leq 1\}\\
            \end{align*}
            (it's the 2 norm ball)
            \begin{align*}
                f: X\rightarrow X\\
                g: X\rightarrow X\\
            \end{align*}
            Then,
            \begin{align*}
                H&: X\times I \rightarrow X\\
                H((x,y),t)&=(1-t)x, (1-t)y)\\
            \end{align*}
            is a homotopy relative to $(0,0)$.
    \end{itemize}
        \begin{theorem} If $f:X\rightarrow Y$ and $g:X\rightarrow Y$ are
            homotopic relative to $x_0$, then:
                \begin{align*}
                    f_{*}&:\pi_1(X,x_0)\rightarrow \pi_1(Y,y_0)\\
                    g_{*}&:\pi_1(X,x_0)\rightarrow \pi_1(Y,y_0)\\
                \end{align*}
                are the same homomorphism.
        \end{theorem}
