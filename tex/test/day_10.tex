\section{Day 10}
    \subsubsection{$\pi_1(S^1)$, continued:}
    \begin{itemize}
        \item \underline{Recap:}
            \begin{align*}
                p&: \R \rightarrow S^{1}\\
                p(x)&=(\cos(2\pi x), \sin(2\pi x))\\
            \end{align*}
            Then there exists a function,
            \begin{align*}
                \phi &: \pi_1(S^{1}, b)\rightarrow p^{-1}(b)\\
                \phi([f])&=\tilde{f}(1)\\
            \end{align*}
            Where $\tilde{f}$ is the lift of $f$ to $\R$ starting at $0$.\\
            E.g.,(draw that spiraleboye)
            \begin{align*}
                \phi&([\text{loop once counterclockwise}])=1\\
                \phi&([\text{loop twice counterclockwise}])=2\\
                \phi&([\text{loop once clockwise}])=-1\\
            \end{align*}
            The fact that there exists a unique lift, $\tilde{f}$ of any $f$ is a feature
            of covering maps.\\
            In fact,
            \begin{align*}
                p^{-1}(b)=\Z\\
            \end{align*}
            and,\\
        \item \underline{Claim:} $\phi: \pi_1(S^1, b)\rightarrow \Z$ is a bijection.
            \begin{proof}
                \begin{enumerate}
                    \item \underline{Surjective:} Given $c\in\Z$, choose a path, $\alpha: I \rightarrow \R$, from
                        $0$ to $c$ in $\R$. Then let, $f: I \rightarrow S^1$ be $f=p\circ \alpha$\\
                        Then f is a loop in $S^1$ based at $b=(1,0)$ because
                        \begin{align*}
                            f(0)=p(\alpha(0))=p(0)=(1,0)\\
                            f(1)=p(\alpha(1))=p(c)=(1,0)\\
                        \end{align*}
                        And, $\tilde{f}=\alpha$ because $p\circ \tilde{f}=p\circ \alpha = f$. Thus,
                        \begin{align*}
                            \phi([f])=\tilde{f}(1)=\alpha(1)=c\\
                        \end{align*}
                    \item \underline{Injective:} Suppose,
                        \begin{align*}
                            \phi([f])=\phi([g])\\
                            \implies \tilde{f}(1)=\tilde{g}(1)\\
                        \end{align*}
                        Then, $\tilde{f}$ and $\tilde{g}$ are two paths in $\R$, that both start at 0
                        and both end at the same point.\\
                        $\Rightarrow$ (courtesy of homework 2) $\tilde{f}\cong_{p}\tilde{g}$ (because $\R$ is simply
                        connected)\\
                        $\Rightarrow$ $p\circ H$ is a path homotopy from $p\circ \tilde{f}$ to $p\circ \tilde{g}$.\\
                        $\Rightarrow f\cong_{p}g$\\
                        $\Rightarrow [f]=[g]\in \pi_1(S^1, b)$\\
                \end{enumerate}
            \end{proof}
        \item \underline{Claim:} $\phi$ is a group homomorphism ( thus, an isomorphism ).\\
            \begin{proof}
                Let $[f], [g]\in \pi \pi_1(S^1,b)$, we want to show that,
                $\phi([f]*[g])=\phi([f])+\phi([g])$\\
                By definition,
                \begin{align*}
                    \phi([f]*[g])=\phi([f*g])=\tilde{f*g(1)}\\
                \end{align*}
                What is $\tilde{f*g}$? By definition $\tilde{f*g}$ is the lift of $f*g$
                starting at $0$ and,
                \begin{align*}
                    \tilde{f}=\text{lift of $f$ starting at 0 ending at some $n$}\\
                    \tilde{g}=\text{lift of $g$ starting at 0 ending at some $m$}\\
                \end{align*}
                %Note some drawing.
                So, $\tilde{f}*\tilde{g}$ doesn't make sense, but let:
                \begin{align*}
                    \tilde{g}^{\prime}&=\text{``shift $\tilde{g}$ by n ''}\\
                    \text{i.e., }\tilde{g}^{'}&=g(s)+n\\
                \end{align*}
                Sow notice that $\tilde{f}*\tilde{g}^{'}$ now makes sense, and $\tilde{g}^{'}$ is a
                lift of $g$, because:
                \begin{align*}
                    (p\circ\tilde{g}^{'})(s)&=p(\tilde{g}(s))\\
                    &=p(\tilde{g}(s)+n)\\
                    &=p(\tilde{g}(s))\\
                \end{align*}
                because $p(x+n)=p(x),\ \forall n\in\Z$
                \begin{align*}
                    &=(p\circ\tilde{g})(s)\\
                    &=g(s)\\
                \end{align*}
                Thus, $\tilde{f}*\tilde{g}^{'}$ is a lift of $f*g$ starting at $0$\\
                \begin{align*}
                    \implies \tilde{f}*\tilde{g}&=\tilde{f*g}\\
                    \tilde{f*g}(1)&=\tilde{f}*\tilde{g}\\
                    &=\text{endpoint of $\tilde{g}^{\prime}$}\\
                    &=\tilde{g}(1)+n\\
                    &=m+n\\
                \end{align*}
                This shows that
                \begin{align*}
                    \phi([f]*[g])&=m+n\\
                    &=\tilde{f}(1)+\tilde{g}(1)\\
                    &=\phi([f])+\phi([g])\\
                \end{align*}
            \end{proof}
        \item \underline{We want:}
            \begin{align*}
                X\cong Y\implies \pi_1(X,x)\cong\pi_1(Y,y)\\
                \text{(X is homeomorphic to Y)}\\
            \end{align*}
            The big tool we'll use to do that is the tool from the second homework
            about maps between spaces being homomorphisms. That's for next time!
    \end{itemize}
