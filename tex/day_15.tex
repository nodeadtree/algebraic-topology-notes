\documentclass[../notes.tex]{subfiles}
\begin{document}
\section{Day 15}
\subsection{Continuation and finish of Day 14}
\begin{itemize}
    \item A space with the same $\pi_1$ as ``B''
    \item \underline{Ex:}
        \begin{align*}
            X &= \R^2\setminus\{p,q\}\\
            p &= (-1,0)\\
            q &= (1,0)\\
        \end{align*}
        To see that $\pi_1(X) \cong \pi(\infty)$ (where infinity isn't actually infinity,
        but two circles joined together to look like a figure eight.), we can construct a 
        deformation retraction of $X$ onto,
        \[
            A=\{(x+1)^2+y^2=1\}\cup \{(x-1)^2+y^2=1\}
        \]
        Pictorially, refer to the picture taken in class.
        \begin{enumerate}
            \item Deformation retract $X$ onto a closed disk of radius 2, centered
                at (0,0)
            \item Then Deformation retract onto a union of two closed disks vertically, 
                again, refer to the picture.
        \end{enumerate}
    \item
        \underline{Ex:}
        \[
            \theta=\{x^2+y^2=1\}\cup\{(x,0)| -1 \leq x \leq 1\}\\
        \]
        Oh yeah, another picture. To see that $\pi_1(\theta)\cong\pi_{1}(\infty)$,
        (again using infinity in lieu of the double circle figure eight)
        we can construct a deformtaion retraction of $\R^2\setminus\{p^{\prime},q^{\prime}\}$
        onto $\theta$ where $p^{\prime}=(0,\frac{1}{2})$ and $q^{\prime}=(0,-\frac{1}{2})$
    \item \underline{Observation:} This shows that,
        \[
            \pi_1(\infty)=\pi_1(\theta)\\
        \]
        because both of them are isomorphic to $\pi_1(\R\setminus\{\text{two points}\})$
        But neither $\infty$ nor $\theta$ are deformation retracts of eachother.
    \item They're related by a more general relationship, that of homotopy equivalence.
\end{itemize}
\subsubsection{Homotopy Equivalence}
\begin{definition}
    A continuous map,
    \[
        f: X\rightarrow Y\\
    \]
    is called a \underline{homotopy equivalence} if there exists a $g:X\rightarrow Y$ such
    that $f\circ g\cong id_Y$, and $g\circ f \cong id_X$, with our equivalency being
    homotopic to.
\end{definition}
\begin{itemize}
    \item \underline{Goals:} A homotopy equivalence induces an $\cong$ on $\pi_1$.
    \item Any deformation retraction ``yields'' a homotopy equivalence, but homotopy
        equivalence is an EQUIVALENCE relation. Sick.
    \item \underline{Ex:}
        \begin{align*}
            X=\R^2\\
            A=\{0,0\}\\
        \end{align*}
        Then $A$ is a deformation retract of $X$.
        \[
            H((x,y),t)=((1-t)x,(1-t)y)
        \]
        But, $A$ is \underline{not} homeomorphic to $X$
    \item \underline{Ex:}
        Wow, yet another picture. Wonderful. Refer to the appropriate photograph.
        Closed disks in $R^2$, then $A$ is a deformation retraction of $X$, and also
        $A$ is homeomorphic to $X$. Look at the picture ya doink.\\
        $X$ is homeomorphic to $Y$ implies that $\pi_1(X)\cong\pi_1(Y)$, but the
        converse is not true, e.g.:
        $X$ is a deformation retraction of $Y$ implies that $\pi_1(X)\cong \pi_1(Y)$
        But not converseley, e.g:
        $X$ is homotopy equivalent to $Y$ implies $\pi_1(X)\cong \pi_1(Y)$.\\
        None of these are conversely true. Wonderful! That was confusing.
\end{itemize}
\end{document}
