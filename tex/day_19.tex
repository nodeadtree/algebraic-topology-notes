\documentclass[../notes.tex]{subfiles}
\begin{document}
    \section{Day 19}
        \subsection{Calculating $\pi_1$ piecewise}
        \begin{itemize}
            \item \underline{Goal:} To calculate $\pi_1(X)$ with 
                $\pi_1(\text{``pieces of X''})$
            \item The ultimate theorem we'll prove for this is the 
                \underline{Van Kampen Theorem}, which will say,
                Let,
                \[
                    X=U\cup V
                \]
                Where $U$ and $V$ are path connected open subsets of X. Furthermore
                their intersection is \textit{also} path connected. Then,
                \[
                    \pi_1(X)=\pi_1(U)?\pi_1(V)\\
                \]
                Note that the ? takes the place of some as yet undefined operation,
                and it will depend not only on $\pi_1(U)$ or $\pi_1(V)$, but their
                interaction.
        \end{itemize}
        \subsection{Free product of groups}
        \begin{definition}
            A \underline{word} in $G$ and $H$ is a string of symbols,
            \[
                a_1,a_2,a_3,\dots,a_n
            \]
            where $a_i$ is either an element of $G$ or an element of $H$
        \end{definition}
        \begin{itemize}
            \item
                \begin{align*}
                    \underbrace{G = \{a^0,a^1,a^2,a^3\}}_{\text{
                            group under multiplication if we declare $a^4=a^0$
                        }
                    }\\
                    \underbrace{H = \{b^0,b^1,b^2\}}_{\text{
                            group under multiplication if we declare $b^3=b^0$
                        }
                    }\\
                \end{align*}
                An example of a word in $G$ and $H$ is 
                \[
                    a^1a^1b^2a^1b^0b^1
                \]
                Thus far, this is a different word from
                \[
                    a^2b^2a^1b^1
                \]
                But we'd like for them to be equivalent. To achieve
                this end, we define two reducing operations on 
                $\{\text{words in $G$ and $H$}\}$
                \begin{enumerate}
                    \item If a word contains a copy of $1_g$, the identity
                        of $G$ or $1_h$ remove that symbol from the word.
                    \item If a word contains two consecutive terms
                        from the same group, replace them with their product.
                \end{enumerate}
                This allows for these two things to be set equal by a sequence
                of the reducing operations.
                \begin{align*}
                    a^1a^1b^2a^1b^0b^1\\
                    a^2b^2a^1b^0b^1\\
                    a^2b^2a^1b^1\\
                \end{align*}
        \end{itemize}
        \begin{definition}
            A word is called \underline{reduced} if no further reducing operations
            can be applied to it.
        \end{definition}
        \begin{itemize}
            \item \underline{Ex:} $a^1b^2a^1$ is the reduced form of $a^1a^{-1}a^1b^1b^1a^1$.
            \item \underline{Observation:} Any word can be converted via a 
                sequence of reducing operations to a unique reduced word.\\
        \end{itemize}
        \begin{definition}
            Let $H$ and $G$ be any groups. Their \underline{free product}
            \[
                G*H=\{\text{reduced words in $G$ and $H$}\}
            \]
            This is a group under the operation of concatenation followed
            by reduction.
            Note that $G*H$ is an infinite non-abelian group.
        \end{definition}
\end{document}
