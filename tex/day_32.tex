\documentclass[../notes.tex]{subfiles}
\begin{document}
\section{Day 32}
\subsection{Singular Homology Goals}
\begin{itemize}
    \item (Eventually) version of the Van Kampen theorem,
    \item Prove that if $X\simeq Y$(homotopy equivalent), then
        $H_n(X)\cong H_n(Y),\ \forall n$
\end{itemize}
\subsection{Homological Algebra}
\begin{definition}
    A \underline{chain complex} is a collection of abelian groups.
    \[
        \dots,C_2,C_1,C_0,C_{-1},C_{-2},\dots
    \]
    and group homomorphisms
    \[
        \partial_n:C_n\rightarrow C_{n-1}\\
    \]
    Such that
    \[
        \partial_{n-1}\circ\partial_n=0
    \]
    \begin{center}
        \begin{tikzcd}
            C_{\cdot}=(\dots \arrow[r, "\partial_3"] &
            C_2 \arrow[r, "\partial_2"] &
            C_1 \arrow[r, "\partial_1"] &
            C_0 \arrow[r, "\partial_0"] &
            C_{-1} \arrow[r, "\partial_{-1}"] &
            C_{-2} \arrow[r, "\partial_{-2}"] &
            \dots)
        \end{tikzcd}
    \end{center}
\end{definition}
\begin{itemize}
    \item \underline{Main Example:}\\
        \begin{align*}
            C_n&=C_n(X)\\
            &=\{\text{$\Z$-linear combinations of n-simplices in $X$}\}\\
            &\text{or}\\
            C_n&=C_n^{\Delta}(X)\\
            &=\{\underbrace{\text{$\Z$-linear combinations of the $n$-simplices in $S_n$}}_{
                \text{(particular $\Delta$-complex str)}
            }\}\\
            &\text{or}\\
            C_n&=C_n(K),\ \text{Where $K$ is a simplicial complex}\\
        \end{align*}
\end{itemize}
\begin{definition}
    If $C_{\cdot}$ is any chain complex, it's \underline{$n^{\text{th}}$-homology group} is
    \[
        H_n(C_{\cdot}):=\frac{ker(\partial_n)}{im(\partial_{n+1})}
    \]
\end{definition}
\begin{itemize}
    \item \underline{Note:} This makes sense because $im(\partial_{n+1})\subseteq ker
        (\partial_n)$, which follos from $\partial_n\circ\partial_{n+1}=0$.
\end{itemize}
\begin{definition}
    If $C_{\cdot}$ and $D_{\cdot}$ are chain compexes, then a \underline{chain map} from
    $C_{\cdot}$ to $C_{\cdot}$ to $D_{\cdot}$ is a collection of homomorphisms
    \[
        f_n:C_n\rightarrow D_n
    \]
    Such that,
    \[
        \partial_n^D\circ f_n=f_{n-1}\circ\partial_n^C,\ \forall n\in \Z
    \]
    \newpage
    Graphically,
    \begin{center}
        \begin{tikzcd}
                \dots \arrow[r, "\partial_3"] &
                D_2 \arrow[r, "\partial_2"] \arrow[d, "f_2"]&
                D_1 \arrow[r, "\partial_1"] \arrow[d, "f_1"]&
                D_0 \arrow[r, "\partial_0"] \arrow[d, "f_0"]&
                D_{-1} \arrow[r, "\partial_{-1}"] \arrow[d, "f_{-1}"]&
                D_{-2} \arrow[r, "\partial_{-2}"] \arrow[d, "f_{-2}"]&\dots\\
                \dots \arrow[r, "\partial_3"] &
                C_2 \arrow[r, "\partial_2"] &
                C_1 \arrow[r, "\partial_1"] &
                C_0 \arrow[r, "\partial_0"] &
                C_{-1} \arrow[r, "\partial_{-1}"] &
                C_{-2} \arrow[r, "\partial_{-2}"] &\dots
        \end{tikzcd}
    \end{center}
    Such that this commutes.
\end{definition}
\begin{itemize}
    \item If $X$ and $Y$ are topological spaces and $g:X\rightarrow Y$ is a continuous
        function, then
        \begin{align*}
            g_{\#}&:C_n(X)\rightarrow C_n(Y)\\
            g_{\#}(\alpha_1\sigma_1+\dots+\alpha_r\sigma_r)&=
            a_1(g\circ\sigma_1)+\cdots a_r(g\circ\sigma_r)\\
        \end{align*}
        is a chain map because we checked $\partial_n^Y\circ g_{\#}=g_{\#}\circ\partial_n^X$
    \item \underline{Exercise:} If $f_{\cdot}:C_{\cdot}\rightarrow D_{\cdot}$ is any
        chain map, then
        \begin{align*}
            f_*:H_n(C_{\cdot})\rightarrow H_n(D_{\cdot})\\
            H_n(C_{\cdot})=\frac{ker(\partial_n^C)}{im(\partial_{n+1}^C)}\\
            H_n(D_{\cdot})=\frac{ker(\partial_n^D)}{im(\partial_{n+1}^D)}\\
            f_{*}([\alpha])=[f_n(\alpha)]\\
        \end{align*}
        Where,
        \begin{align*}
            \alpha \in ker(\partial_n^C)\subseteq C_n\\
            f_n(\alpha)\in D_n\\
        \end{align*}
        is well defined. (Proof is exactly what we did to show that 
        $g_{*}:H_n(X)\rightarrow H_n(Y)$ is well defined
    \item \underline{Question:} When do two different chain maps,
        \begin{align*}
            f_{\cdot}:C_{\cdot}\rightarrow D_{\cdot}\\
            g_{\cdot}:C_{\cdot}\rightarrow D_{\cdot}\\
        \end{align*}
        induce the \underline{same} homomorphism
        \[
            f_{*}=g_{*}:H_n(C_{\cdot})\rightarrow H_n(D_{\cdot})
        \]
        (\underline{Structurally:} if $X$ and $Y$ are homotopy equivalent, then there
        exists
        \begin{center}
            \begin{tikzcd}
                X \arrow[r, bend left=50, "f"]& Y\arrow[l, bend left=50, "g"]
            \end{tikzcd}
        \end{center}
        such that $f \circ g\simeq id_Y$ and $g\circ f\simeq id_X$. For $\pi_1$ we 
        ``showed'' $(f\circ g)_{*}=(id_Y)_*$ and $(g\circ f)_{*}=(id_X)_*$, thus
        $f_{*}$ and $g_{*}$ were inverse isomorphisms.
        \begin{center}
            \begin{tikzcd}
                \pi_1(X) \arrow[r, bend left=50, "f_*"]& \pi_1(Y)
                \arrow[l, bend left=50, "g_*"]
            \end{tikzcd}
        \end{center}
        To do this argument for $H_n$ instead of $\pi_1$ we'll need to know that homotopic 
        maps between spaces induce the \underline{same} homomorphism on homology.)
\end{itemize}
\begin{definition}
    A \underline{chain homotopy} $f_{\cdot}:C_{\cdot}\rightarrow D_{\cdot}$ to
    $g_{\cdot}:C_{\cdot}\rightarrow D_{\cdot}$ is a collectopn of homomorphisms
    \[
        h_n:C_n\rightarrow D_{n+1},\ \forall n\in \Z
    \]
    such that
    \[
        \partial_{n+1}^D\circ h_n+h_n\circ \partial_{n+1}^C=f_n-g_n
    \]
    \begin{center}
        \begin{tikzcd}
                \dots \arrow[r, "\partial_3"] &
                D_2 \arrow[r, "\partial_2"] \arrow[d, "f_2" description, bend right=20, color=orange]
                    \arrow[d, "g_2" description, bend left=20, color=blue]&
                D_1 \arrow[r, "\partial_1"] \arrow[d, "f_1" description, bend right=20, color=orange]
                    \arrow[d, "g_1" description, bend left=20, color=blue]\arrow[dl, "h_1" description, color=violet]&
                D_0 \arrow[r, "\partial_0"] \arrow[d, "f_0" description, bend right=20, color=orange]
                \arrow[d, "g_0" description, bend left=20, color=blue]\arrow[dl, "h_0" description, color=violet]&\dots\\
                \dots \arrow[r, "\partial_3"] &
                C_2 \arrow[r, "\partial_2"] &
                C_1 \arrow[r, "\partial_1"] &
                C_0 \arrow[r, "\partial_0"] & \dots
        \end{tikzcd}
    \end{center}
\end{definition}
\begin{lemma}
    If there exists a chain homotopy from $f_{\cdot}$ and $g_{\cdot}$ then $f_{*}=g_{*}$
\end{lemma}
\begin{proof}
    Both $f_{*}$ and $g_{*}$ are homomorphisms,
    \begin{align*}
        H_n(C_{\cdot})\rightarrow H_n (D_{\cdot}) \\
        H_n(C_{\cdot})=\frac{ker(\partial_n^C)}{im(\partial_{n+1}^C)}
    \end{align*}
    Let $[\alpha]\in H_n(C)$
    \begin{align*}
        f_{*}([\alpha])-g_{*}([\alpha])&=[f_n(\alpha)]-[g_n(\alpha)]\\
        &=[f_n(\alpha)-g_n(\alpha)]\\
        &=[\partial_{n+1}^D\circ h_n(\alpha)]+[h_{n-1}\circ\partial_n^C(\alpha)]\\
        &=0
    \end{align*}
\end{proof}
\end{document}

