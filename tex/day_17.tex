\subsection{Day 17}
\subsubsection{Homotopy Equivalence}
\begin{itemize}
    \item \underline{Goal:} A homotopy equivalence induces an isomorphism on $\pi_1$
    \item This follows from:
\end{itemize}
\begin{theorem}
    If $f: X\rightarrow Y$ and $g: X\rightarrow Y$ are continuous, $f\cong g$ (homotopic),
    \[
        f(x_0)=y_0,\ g(x_0)=y_1
    \]
    Then there exists a path, $\alpha$ from $y_0$ to $y_1$ such that 
    $g_* = \hat{\alpha}\circ f_*$.\\
    \underline{Schematically:}
    \[
        \underbrace{\pi_1(X,x_0)\xrightarrow{f_*}\pi_1(Y,y_0)
            \xrightarrow{\hat{{\alpha}}}\pi_1{Y,y_1}}_{g_*}
    \]
\end{theorem}
    \begin{proof}
        Let
        \[
            H:X\times I \rightarrow Y
        \]
        be a homotopy from $f$ to $g$ (I.e, $h_t:X \rightarrow Y,\ \forall t\in I$).
        Let,
        \begin{align*}
            \alpha: I \rightarrow Y\\
            \alpha(t)=h_t(x_0)\\
        \end{align*}
        Note that we want,
        \begin{align*}
            \forall [\gamma] &\in \pi_1(X,x_0):\\
            g_*([\gamma])&=\hat\circ f_*([\gamma])\\
            \iff [g\circ \gamma]&=[\overline{\alpha}*(f\circ \gamma)*\alpha]\\
            \iff g\circ \gamma&\cong \overline{\alpha}*(f\circ \gamma)*\alpha\\
        \end{align*}
        We'll prove these are path homotopic by interpolating between them by the following
        loops: (There's some drawing that goes here)
        Explicitly, let,
        \begin{align*}
            \beta_{t}&: I \rightarrow Y\\
            \beta_{t}(s)&=\overline{\alpha}((1-t)s)\\
        \end{align*}
        Then,
        \begin{align*}
            \beta_{0}&=\overline{\alpha}\\
            \beta_{1}&=e_{y_1}\\
            \beta_{t}&=\text{path from $y_1$ fo $\alpha(t)$}\\
        \end{align*}
        Now, define the followin loop at $y_1$:
        \begin{align*}
            \beta_{t}*(h_t\circ \gamma)* \overline{\beta_{t}}\\
        \end{align*}
        This is:
        \begin{enumerate}
            \item When $t=0$:
                \[
                    \beta_{0}*(h_0\circ \gamma)*\overline{\beta_0}
                    =\overline{\alpha}*(f\circ \gamma)*\alpha
                \]
                (this is the green loop from the hard to see picture)
            \item When $t=1$:
                \[
                    \beta_{1}*(h_1\circ \gamma)*\overline{\beta_1}
                    =e_{y_{1}}*(f\circ \gamma)*e_{y_{1}}
                \]
        \end{enumerate}
        Thus, $\beta_{t}*(h_t\circ \gamma)*\overline{\beta_t}$ give a path homotopy,
        \[
                    \overline{\alpha}*(f\circ \gamma)*\alpha
                    \cong_p e_{y_{1}}*(f\circ \gamma)*e_{y_{1}}
                    \cong_p g \circ \gamma\\
        \]
    \end{proof}
    \begin{corollary}
        If $f: X\rightarrow Y$ is a homotopy equivalence
        (recall that this means there exists a $g:Y\rightarrow X$ such that 
        $f\circ g\congid_Y$ and $g\circ f\cong id_X). Then,
        \[
            f_*: \pi_1(X,x_0)\rightarrow \pi_1(Y,y_0)
        \]
        is an isomorphism.
        \begin{proof}
            We know,
            \begin{align*}
                g\circ f \cong id_X\\
                \xRightarrow[\text{theorem}]\ (g\circ f)_*
                =\hat{\alpha}\circ (id_X)_*, \text{ for some path $\alpha $ }\\
                \Rightarrow g_*\circ f_* = \hat{\alpha}\\
            \end{align*}
            and similarly,
            \begin{align*}
                f\circ g\cong id_Y\\
                (f\circ g) = \hat{\beta}\circ (id_Y)_* \text{ for some path in $\beta $ }\\
                \rightarrow f_* \circ g_* = \hat{\beta}\\
            \end{align*}
            Thus, if $f(x_0)=y_0$, $g(y_0)=x_1$, $f(x)=y_1$:\\
            \begin{center}
                \begin{tikzcd}
                    \pi_1(X,x_0) \arrow[d, "\hat{\alpha}"] \arrow[r, "f_*"]& \pi_1(Y,y_0) 
                    \arrow[ld, "g_*"'] \arrow[d, "\hat{\beta}"]\\
                    \pi_1(X,x_1)\arrow[r, "f_*"] &\pi_1(Y,y_1)\\
                \end{tikzcd}\\
            \end{center}
            (Some fuckin arrow diagram Ah fuck)
            Therefore,
            \begin{align*}
                g_*\circ f_* = \hat{\alpha}, \text{ an isomorphism! }\\
                \Rightarrow g_* \text{ is surjective}\\
            \end{align*}
            And similarly, because
            \begin{align*}
                f_*\circ g_* = \hat{\beta} \text{ (an isomorphism!) }\\
                \rightarrow g_* \text{ is injective}
        \end{proof}
    \end{corollary}
    \end{theorem}
