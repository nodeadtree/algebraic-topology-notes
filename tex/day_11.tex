\section{Day 11}
    \subsubsection{Examining the group structure of $*$ functions}
    \begin{enumerate}
        \item Note that this Friday, office hours will be at 3-4pm.
        \item \underline{We want:} If $X\cong Y$, then $\pi_1(X,x)\cong \pi_1(Y,y)$,
            or that, if two spaces are homeomorphic, then their fundamental groups are isomorphic. We
            will explore the tools used to show this in this lecture. From homework 2, we get the following
            definition
        \item 
            \begin{definition}
                Let $\varphi: X\rightarrow Y$, be a continuous
                map, then the \underline{homomorphism induced by $\varphi$} is:
                \begin{align*}
                    \varphi_{*}: \pi_1(X,x)\rightarrow \pi_1(Y,y)\\
                    \varphi_{*}([f])=[\varphi\circ f]\\
                \end{align*}
                See the picture of the picture drawn on the board, make a drawyboye.
            \item \underline{Lemma:} (this is referred to lemma 1)If
                \begin{align*}
                    X\rightarrow^{\varphi}Y\rightarrow^{\psi}Z\\
                \end{align*}
                Where $\varphi$ and $\psi$ are both continuous, then,
                \begin{align*}
                    (\psi \circ \varphi)_{*}=\psi \circ_{*} \varphi_{*}\\
                \end{align*}
                Additionally, (This is referred to as lemma 2)
                \begin{align*}
                    {id}_{*}=id\\
                \end{align*}
                (or that given the $id: X\rightarrow Y$, the induced homomorphism,
                $\pi_{1}(X,x)\rightarrow \pi_{1}(Y,y)$ is the identity)
            \end{definition}
        \item \begin{proof}
                    Firstly, Both sides are homomorphisms
                        \begin{align*}
                            \pi_1(X,x)\rightarrow\pi_1(Z,(\psi\circ\varphi)(x))\\
                        \end{align*}
                    Given any $[f]\in\pi_1(X,x)$:
                        \begin{align*}
                            (\psi\circ\varphi)_{*}([f])=[(\psi\circ\varphi)\circ f]\\
                            =[\psi\circ(\varphi\circ f)]\\
                            =\psi_{*}[\varphi\circ f]\\
                            =\psi_{*}(\varphi_{*}(f))\\
                            =(\psi_{*}\circ\varphi_{*})([f])\\
                        \end{align*}
                    Given any $[f]\in\pi_1(X,x)$:
                        \begin{align*}
                            id_{*}([f])=[id\circ f]\\
                            =[f]\\
                        \end{align*}
        \end{proof}
        \item 
            \begin{theorem} if $\varphi: X\rightarrow Y$ is a homeomorphism,
            then $\varphi_{*}: \pi_{1}(X,x)\rightarrow\pi_{1}(Y,y)$ is an isomorphism.
            \end{theorem}
                \begin{proof}
                    We already know that $\varphi_{*}$ is a homomorphism, to prove that it's
                    a bijection, we'll find an inverse to $\varphi_{*}$. Claim that,
                    \begin{align*}
                        (\varphi)_{*}:\pi_1(Y,\varphi(x))\rightarrow\pi_1(X,x)\\
                    \end{align*}
                    is the inverse to $\varphi_{*}$.\\
                    (Note that this is doable, because $\varphi$ is a homeomorphism, $\varphi^{-1}:Y\rightarrow X$ exists, and is continuous)\\
                    To check this:
                    \begin{align*}
                        \varphi_{*}\circ (\varphi^{-1})\\
                        =(\varphi\circ\varphi^{-1}),\ \text{by lemma 1 shown today}\\
                        =id_{*},\ \text{by definition of $\varphi^{-1}$ (identity on y)}\\
                        =id,\ \text{by lemma 2 shown today (identity on x)}\\
                        (\varphi^{-1})_{*}\circ\varphi_{*}=(\varphi^{-1}\circ\varphi)_{*}=id_{*}=id\\
                    \end{align*}
                    This by definition means $\varphi_{*}$ and $(\varphi^{-1})_{*}$ are inverse functions.
                    Additionally, this small red box has made it onto the board, for clarification.
                    \begin{align*}
                        id_x: X \rightarrow Y\\
                        id_{\pi_1(X,x)}:\pi_1(X,x)\rightarrow\pi_1(X,x)\\
                        \text{Lemma: }(id_{x})_{*}id_{\pi_1(X,x)}\\
                    \end{align*}
                \end{proof}
        \item
            This ends up proving that,
            \begin{align*}
                X\cong Y\implies\pi_1(X,x)\cong\pi_1(Y,\varphi(x))\\
            \end{align*}
            But, non-homeomorphic spaces \underline{can} have isomorphic $\pi_{1}$\\
            \underline{Ex:}
            \begin{align*}
                X=.\\
                Y=\R^2\\
            \end{align*}
            These are not homeomorphic, clearly X is compact and Y isn't, but their fundamental groups are
            isomorphic, since the fundamental group of $X$ is just $\{1\}$, and clearly this is also true
            about $\R^2$
        \item So, given $X$ and $Y$, how can we tell if $\pi_1(X)\cong\pi_1(Y)$?
    \end{enumerate}
        \subsubsection{Homotopy of Maps:}}
            \begin{definition} Let $f: X\rightarrow Y$ and $g: X\rightarrow Y$ be continuous functions.
            Then a \underline{homotopy} from $f$ to $g$ is a continuous function,
            \begin{align*}
                H: X\times I\rightarrow Y\\
            \end{align*}
            such that,
            \begin{align*}
                H(x,0)=f(x),\ \forall x\in X\\
                H(x,1)=f(x),\ \forall x\in X\\
            \end{align*}
            Our goal is to make remark about the lower star versions of these maps, given their being homotopic.
            \end{definition}
