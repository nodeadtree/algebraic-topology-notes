\documentclass[../notes.tex]{subfiles}
\begin{document}
\section{Day 23}
\subsection{Conclusion of Van Kampen}
\begin{itemize}
    \item \underline{Recall:}
        Proving the following,
        \[
            \pi_1(X)\cong \frac{\pi_1(U)*\pi_1(V)}{N}
        \]
        Where,
        \[
            N=\text{Smallest normal subgroup containing $i_{U*}(h)i_{V*}^{-1}(h)\forall h\in\pi_1(U\cap V)$}
        \]
        Using
        \[
            j:\pi_1(U)*\pi_1(V)\rightarrow \pi_1(X)
        \]
        induced by the homomorphisms
        \begin{align*}
            j_{U*}:\pi_1(U)\rightarrow \pi_1(X)
            j_{V*}:\pi_1(V)\rightarrow \pi_1(X)
        \end{align*}
        So far, we proved
        \begin{enumerate}
            \item j is surjective
            \item $N\subseteq ker(j)$
        \end{enumerate}
        For step 3, we need to show that $ker(j)\subseteq N$, then we're
        finished by the first isomorphism theorem.
    \item \underline{Idea of proof} Let,
        \[
            w=[w_1][w_2]...[w_k]\in ker(j)
        \]
        This means if we define
        \[
            f_i = \begin{cases}
                j_U*w_i & \text{ if} [w_i]\in \pi_1(U)\\
                j_V*w_i & \text{ if} [w_i]\in \pi_1(V)\\
            \end{cases}
        \]
        Then,
        \begin{align*}
            \underbrace{[f_1*f_2*\dots*f_k]}_{\text{this is $j(w)$}}=[e_{x_0}]\\
            \iff f_1*\dots*f_k\cong_{p}e_{x_0}\\
        \end{align*}
        We want, $w\in N$
        First, we define a ``move'' on words $w\in \pi_1(U)*\pi_1(V)$:
        \begin{itemize}
            \item If $[w_i]\in \pi_1(U)$, but $w_i$ actually lives in $U\cap V$, 
                then view $[w_i]\in\pi_1(V)$ instead. (or vice versa)
        \end{itemize}
        This amounts to multiplying $w$ by an element of $N$! Why though?
        ``[w_i]\in \pi_1(U) but $w_i$ actually lies in $U\cap V$''
        means,
        \[
            [w_i]=i_{U*}([\tilde{w}_i])\text{ for some $[\tilde{w}_i]\in \pi_1(U\cap V)$ }
        \]
        ``view $[w_i]\in \pi_1(V)$ instead'', means to replace $[w_i]$ by 
        $[w^{\prime}=i_{V*}([\tilde{w}_i])]$, Thus, a move means,
        \begin{align*}
            w=[w_1][w_2]\dots\underbrace{[w_i]}_{=i_{U*}([\tilde{w}_i])}\dots[w_k]\rightsquigarrow 
            [w^{\prime}_1][w^{\prime}_2]\dots\underbrace{[w^{\prime}_i]}_{=i_{V*}([\tilde{w}_i])}\dots[w^{\prime}_k]
        \end{align*}
        To get from LHS to RHS, multiply $w$ by
        \[
            ([w_{i+1}]\dots[w_{k}])i_{U*}^{-1}([\tilde{w}_i])i_{V*}([\tilde{w}_i])([w_{i+1}]\dots[w_k])
        \]
        This is an element of $N$.
        From here, we use the path homotopy,
        \[
            f_1*\dots f_k\cong_{p}e_{x_0}
        \]
        to cook up a sequence of moves taking $w$ to $[e_{x_0}]$. This means
        \begin{align*}
            w(\text{something in $N$})\\
            \Rightarrow w\in N\\
        \end{align*}
    \item \underline{Wrapup of $\pi_1$}
        Where do we go from here?
        \begin{itemize}
            \item \underline{Knot theory:} A ``knot'' in $\R^3$ is $K:I\rightarrow \R^3$ such that,
                $K(0)=K(1)$ but $K_{|_{(0,1)}}$ is injective.\\
                That is to say that two knots, $K_0$ and $K_1$ in $\R^3$ are \underline{equivalent} if there exists a family
                of homeomorphisms
                \[
                    h_t:\R^3\rightarrow \R^3\\
                \]
                such that $h_0=id$ and $h_0\circ K_0=K_1$\\
                Turns out that you can use $\pi_1(\R^3\setminus im(K))$ in a sneaky way to
                detect if two knots are equivalent.
            \item \underline{The Fundamental Theorem of Algebra:} This can be proven with the
                fundamental group. Has a proof via $\pi_1$ that involves viewing a polynomial with no roots in $\C$ as
                a map
                \[
                    p: \C \rightarrow \C\{0\}
                \]
                and arguing,
                \[
                    p_{*}([\text{some loop}])\neq 1
                \]
                Which yields a contradiction because $\pi_1(\C)=\{1\}$
            \item Siilar ideas prove Borsuk-Ulam Theorem and Hairy Ball Theorem, and yadda yadda yadda.
        \end{itemize}
    \end{itemize}
\end{document}
