\documentclass[../notes.tex]{subfiles}
\begin{document}
\section{Day 26}
\subsection{Simplicial Homology}
\begin{itemize}
    \item Let the following be so,
        \begin{align*}
            K=\text{Simplicial complex}\\
            K_n=\{\text{ n-simplices of $K$ }\}\\
            C_n(K)=\{\text{ n-chains }\}\\
            =\{\text{$\Z$-linear combinations of n-simplices of $K$}\\
        \end{align*}
    \item Choose an ordering of the vertices, (0-simplices) of K,
        \[
            K_0=\{v_0,v_1,\dots,v_k\}
        \]
    \item Then,
        \begin{align*}
            \partial_n:C_n(K)\rightarrow C_{n-1}(K)\\
            \partial_n(\underbrace{[v_{i_0},v_{i_1},\dots,v_{i_n}])}_{\text{
                    n-simplex spanned by $v_{i_0},v_{i_1},\dots,v_{i_n}$}}
                =
            \sum_{j=0}^n(-1)^j[v_{i_0},v_{i_1},\dots,\hat{v}_{i_j}\dots,v_{i_n}])
        \end{align*}
        Where, $i_0\le_1\dots\le i_n$.
\end{itemize}
\begin{definition}
    The \underline{simplicial chain complex} of $K$ is,
    \[
        \{0\}\rightarrow\dots \rightarrow^{\partial_3}C_2(k)
        \rightarrow^{\partial_2}C_1(k)
        \rightarrow^{\partial_1}C_0(k)
        \rightarrow \{0\}
    \]
    By convention,
    \[
        C_n(K)=\{0\}
    \]
    Whenever $K$ has no n-simplices.
\end{definition}
\begin{definition}
    The \underline{$n^{\text{th}}$ simplicial homology group} of $K$ is,
    \[
        H_n(K)=\frac{ker(\partial_n)}{im(\partial_{n+1})}
    \]
\end{definition}
\begin{itemize}
    \item \underline{Note:} This quotient makes sense because,
        \[
            im(\partial_{n+1})\subseteq ker(\partial_n)
        \]
        or equivalently,
        \[
            \partial_n \circ \partial_{n+1}=0
        \]
\end{itemize}
\subsubsection{Worksheet stuff}
\begin{enumerate}
    \item Simplicial chain complex for worksheet
        \[
            \{0\}\rightarrow C_2(k)
            \rightarrow^{\partial_2}C_1(k)
            \rightarrow^{\partial_1}C_0(k)
            \rightarrow \{0\}
        \]
    \item 
        \begin{align*}
            \partial_1(e_1) = \partial_1([v_1,v_2]) = v_2-v_1\\
            \partial_1(e_2) = \partial_1([v_1,v_3]) = v_3-v_1\\
            \partial_1(e_3) = \partial_1([v_2,v_3]) = v_3-v_2\\
        \end{align*}
    \item 
        \begin{align*}
            \partial_2([v_1,v_2,v_3])=[v_2,v_3]-[v_1,v_3]+[v_1,v_2]\\
            \partial_2([v_1,v_2,v_3])=e_3-e_1+e_2
        \end{align*}\\
        \begin{center}
            \begin{tikzpicture}
                \draw[->] (0,0) node [below left] {$p_1$} --
                          (0,6) node [above]{$p_2$} --
                          (6,0) node [below right] {$p_3$} -- cycle;
                          \draw (3,0) node [below] {$e_1$}\\
                          \draw (0,3) node [left] {$e_2$}\\
                          \draw (3,3) node [above right] {$e_3$}\\
            \end{tikzpicture}
        \end{center}
    \item 
        \begin{align*}
            ker(\partial_2)=\{af_1| ker(af_1)=0\}
        \end{align*}
        Since $\partial_2(f_1)\neq 0$, the only multiple of $f_1$ sent to 0 is $0f_1$
    \item
        \[
            ker(\partial_1)=\{ae_1-ae_2+ae_3|\ a\in \Z\}
        \]
\end{enumerate}
\end{document}
