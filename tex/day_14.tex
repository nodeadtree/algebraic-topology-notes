\documentclass[../notes.tex]{subfiles}
\begin{document}
\section{Day 14}
    \begin{itemize}
        \item Quiz(Midterm) on Monday, whenever that is. Standard Dr. Clader Format. Last
            covered topic on that will be deformation retractions.
        \item Recall from last time:
    \end{itemize}
    \begin{theorem}
        \begin{align*}
            A&\subseteq X\\
            x_0& \in A\\
            H&=\text{ deformation retraction of $X$ onto $A$ }
        \end{align*}
        Recall that H is a homotopy relative to $x_0$
        \begin{align*}
            id&: X\rightarrow X\\
            r&: X \rightarrow X\\
            s.t.\ r(x)\in A&,\ \forall x\in X\\
        \end{align*}
        Then,
        \begin{align*}
            \pi_1(X,x_0)\cong\pi_1(A,x_0)\\
        \end{align*}
        ($r$ is a retraction)
    \end{theorem}
        \begin{proof}
            Consider, $X \leftrightharpoons A$, where
            $X\rightarrow A$ is $s$, the same function as $r$,
            and $A\rightarrow X$ is the inclusion map. Then,
            \begin{align*}
                s\circ i &= id: A\rightarrow A\\
                \implies s_* \circ i_* &= id: \pi_1(A,x_0)\rightarrow\pi_1(A,x_0)\\
            \end{align*}
            In the other order:
            \begin{align*}
                i\circ s = r\cong id\\
            \end{align*}
            Note that $r$ is a homotopy relative to $x_0$, and
            that the next step follows from the theorem from the beginning of
            last class.
            \begin{align*}
                \implies i_* \circ s_* = id: \pi_1(X,x_0)\rightarrow\pi_1(X,x_0)\\
            \end{align*}
            So we have:
            \begin{align*}
                \pi_1(X,x_0)\leftrightharpoons\pi_1(A,x_0)\\
                s_*:\pi_1(X,x_0)\rightarrow\pi_1(A,x_0)\\
                i_*:\pi_1(A,x_0)\rightarrow\pi_1(X,x_0)\\
            \end{align*}
            and we've shown $s_*$ and $i_*$ are inverses, giving
            \begin{align*}
                \pi_1(X,x_0)\cong\pi_1(A,x_0)\\
            \end{align*}
        \end{proof}
\begin{itemize}
    \item \underline{Fun Font Fabtacular}
        Letter fundamental groups.
        \begin{enumerate}
            \item \textbf{C family:} C,E,F,G,H,I,J,K,L,M,N,S,T,U,V,W,X,Y,Z
            \item \textbf{A family:} A,D,O,P,Q,R
            \item \textbf{B family:} B (fuckin loser.)
        \end{enumerate}
        The reason $\pi_1(E)\cong\pi_1(I)$ is that there
        is a deformation retraction.
        \begin{align*}
            H&: E\times I \rightarrow E\\
            H(x,0)&=x\\
            H(x,1)&\in I, \text{ the letter ``I'' }\\
        \end{align*}
        The rest of this was erased, before I could write it down. Ahh damn.\\
        Let's talk about $\pi_1(B)$ though. What is that?
        \begin{enumerate}
            \item It's the same as the fundamental group of a figure 8, because $B\cong \infty$
            \item It's also the same as:
                \begin{align*}
                    \pi_1(\R^2\setminus \{p,q\})\\
                \end{align*}
                where $p$ and $q$ are unequal points in $\R^2$.
            \item Also the same as $\pi_1(\theta)$ (theta is just the letter theta)
                
        \end{enumerate}
\end{itemize}
\end{document}