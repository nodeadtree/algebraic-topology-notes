
\section{Day 1}
\subsection{Syllabus Junk}
    \begin{itemize}
        \item Pictures + Computer are ok so long as they're used for note taking.
        \item Expect for the tests to be at ends of the first third of the class, and the second third of the class.
        \item Theoretically this is a graduate course, and will be switched to 852, rather than remaining as 452.
    \end{itemize}
\subsection{The idea of algebraic topology}
    \begin{itemize}
        \item Given topological spaces $X$ and $Y$, how can we prove that $X$ and $Y$ are or aren't homeomorphic.
        \item To prove $X\cong Y$, we simply exhibit a homeomorphism.\\
            E.g. $(-1,1) \cong \R$, using $f(x) = \frac{x}{1-x^2}$\\
            E.g. $\square \cong \circ$
        \item To prove $X\ncong Y$, we'd find a topological invariant, (connected, compact, Hausdorff,\ldots), that only one has.\\
            E.g. $(0,1) \ncong [0,1]$, here, the closed interval is compact, and the open interval is not.\\
            E.g. $(0,1) \ncong [0,1)$, because,\\
            \begin{align*}
                [0,1)\setminus \{0\} = (0,1) \text{ which is connected, but}\\
                (0,1)\setminus \{\text{any point}\} \text{ is disconnected}\\
            \end{align*}
            Note, with the following exercise, If $X\cong Y$ via a homeomorphism, $\psi : X\rightarrow Y$,
            then $X\setminus\{p\}\cong Y\setminus \{\psi(p)\}$
        \item Show the following. \\
            \begin{align*}
                \R \ncong \R^2\\
            \end{align*}
            Here, we note that $\R \setminus \{0\}$ is disconnected.\\
            Suppose towards contradiction that $\R \cong\R^2$, call the homeomorphism $\phi:\R\rightarrow \R^2$, because
            $\R\setminus \{0\}$, the excercise implies that $\R\setminus \{0\}\cong \R^2\{\phi(0)\}$, and therefore
            $\R^2\setminus\{\phi(0)\}$ is disconnected, but that's just wrong, because $\R^2$ without a single point
            is still connected, rigorously showing this should be done through working with path connectedness. Therefore
            these are not homeomorphic.
            \begin{align*}
                \R^2 \ncong \R^3\\
            \end{align*}
            This was a trick question, we don't actually have any topological properties that we can rely on. If we were
            to attempt to remove a line from $\R^2$, we don't have enough information about what the line is homeomorphic to
            in $\R^3$, which is the major stumbling block.
        \item\underline{The Fundamental Group}
        \item The fundamental group is a waay to associate a topological space $X$ to a group $\pi_1(X)$ so that
            $X\cong Y \Rightarrow \pi_1(X)\cong\pi_2(Y)$. 
        \item We'll be able to use this to prove spaces aren't homeomorphic.\\
            \underline{Ex:} In this course we'l learn the following.
            \begin{align*}
                \pi_1(\R^2\setminus\{(0,0)\})=\Z\\
                \pi_2(\R^3\setminus\{\text{any point}\})=\{1\}\\
                \pi_1(\R^2\setminus\{(0,0)\})\ncong \pi_2(\R^3\setminus\{\text{any point}\})\\
                \R^2\ncong\R^3\\
            \end{align*}
            Using this, we can show that these things are not homeomorphic, which is why we do algebraic topology.
            More powerful tools allow for more results.
        \item Note: It's not true that $\pi_1(X)\cong \pi_2(Y)\Rightarrow X\cong Y$\\
            More generally, algebraic topology is about associating the topological space $X$ 
            with the algebraic object $A(X)$, in such a way that $X\cong Y \Rightarrow A(X)\cong A(Y)$\\
            There's a spectrum though.
            \begin{enumerate}
                \item Easy to compute and says nothing, $A(x)$ is the same for all of $X$
                \item Hard to compute, but says everything, $A(X)\cong A(Y) \iff X\cong Y$
            \end{enumerate}
    \end{itemize}
