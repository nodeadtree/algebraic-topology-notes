\subsection{Day 18}
\subsubsection{Homotopy equivalences, concluded}
Recall,
\begin{definition}
    A continuous function $f:X\rightarrow Y$ is a \underline{homotopy equivalence} if there exists
    a continuous function $g:Y\rightarrow X$ such that $g\circ f\simeq id_x$ and $f\circ g \simeq id_Y$\\

\end{definition}
    \underline{Notation/terminology:}
    We call $g$ a \underline{homotopy inverse} of $f$ if there exists a homotopy equivalence,
    $f:X\rightarrow Y$, we say $X$ and $Y$ are \underline{homotopy equivalent} and we write $X\simeq Y$
    \begin{enumerate}
        \item \underline{Ex:}
            \begin{align*}
                f&: \R^2\setminus\{(0,0)\}\rightarrow S^{1}\\
                f(x,y)&=(\frac{x}{\sqrt{x^2+y^2}},\frac{y}{\sqrt{x^2+y^2}})\\
            \end{align*}
            is a homotopy equivalence with homotopy inverse,
            \begin{align*}
                i&: S^{1}\rightarrow R^2\setminus\{(0,0)\}\\
                i(x,y)&=(x,y)\\
            \end{align*}
            To check these are homotopy inverses:
            \begin{align*}
                f\circ i &= id_{S^1}\\
                (i\circ f)(x,y)&=(\frac{x}{\sqrt{x^2+y^2}},\frac{y}{\sqrt{x^2+y^2}})\\
            \end{align*}
            So a homotopy between $f\circ i$ and $id_{S^{1}}$ is,
            \begin{align*}
                H&:S^{1}\times I\rightarrow S^{1}\\
                H((x,y),t)&=(x,y),\ \forall t\in I\\
            \end{align*}
            A homotopy between $i\circ f$ and $id_{\R^2\setminus\{(0,0)\}}$ is the straight line homotopy. This is
            the deformation retraction.
        \item \underline{Ex:} Let,
            \begin{align*}
                D&=\{(x,y)\in \R^2| x^2+y^2\leq 1\}\\
            \end{align*}
            Then,
            \begin{align*} 
                f&: D\rightarrow \{(0,0)\}\\
                f(x,y)&=(0,0)\\
            \end{align*}
            is a homotopy equivalence with homotopy inverse,
            \begin{align*}
                i&: \{(0,0)\}\rightarrow D\\
                i(0,0)&=(0,0),\ \text{(inclusion)}\\
            \end{align*}
            To check these are homotopy inverses:
            \begin{align*}
                f\circ i &= id_{\{(0,0)\}}\\
                (i\circ f)(x,y)&=(0,0)\\
            \end{align*}
            So, a homotopy betwen $i\circ f: D \rightarrow D$ and $id: D\rightarrow D$ is
            \begin{align*}
                H&: D\times I \rightarrow D\\
                H((x,y), t)&=((1-t)x, (1-t)y)\\
            \end{align*}
            This is the straight-line homotopy and it is the deformation retraction of $D$
            onto $\{(0,0)\}$
        \item \underline{Note:}(HW) Any deformation retraction of $X$ onto $A$
            gives rise to a homotopy equivalence $X\simeq A$\\
            (Note that this side box was created at some point.
            $H&:X\times I \rightarrow X$ a homotopy from $id_X:X\rightarrow X$
            to $r:X\rightarrow X$ such that $r(x)\in A,\ \forall x\in X$)
    \end{enumerate}
    \begin{definition}
        If a topological space $X$ is homotopy equivalent to a single point, we say that $X$ is \underline{contractible}.
    \end{definition}
    \begin{enumerate}
        \item\underline{Ex:} The unit disk, $D$ is retractible.
        \item \underline{Note:} By the theorem from last class,
            \[
                \text{contractible}\Rightarrow \text{simply-connected}
            \]
    \end{enumerate}
    \subsection{Why care about homotopy equivalences?}
    Why do we care about homotopy equivalences instead of just using deformation retractions?
        \begin{enumerate}
            \item Deformation retraction is weirdly asymmetric. $A$ is a deformation
                retraction of $X$ but not vice versa, while homotopy equivalence is
                an equivalence relation (courtesy of HW). The fact that it's symmetric
                \begin{itemize}
                    \item {Ex:}
                        \begin{align*}
                            D\underbrace{\simeq}_{\text{last example}}& \{.\}
                            \underbrace{\simeq}_{\text{handout}} \R\\
                            \Rightarrow D&\simeq \R\\
                            \Rightarrow \pi_1(D)&\cong \pi_1(\R)\\
                        \end{align*}
                    \item {Ex:}
                        Looking at the character $\theta$,
                        \begin{align*}
                            \theta
                            \underbrace{\simeq}_{\text{deformation retraction}}
                            &\R^2\setminus{x,y}
                            \underbrace{\simeq}_{\text{deformation retraction}}
                            \infty\\
                            \Rightarrow \theta &\cong \infty\\
                            \Rightarrow \pi_1(\theta)&\cong \pi_1(\infty)\\
                        \end{align*}
                \end{itemize}
            \item We don't have to worry about base points staying fixed throughout the
                homotopy.
        \end{enumerate}
        

