\documentclass[../notes.tex]{subfiles}
\begin{document}
\section{Day 7}
    \subsubsection{Guest lecturer: Mattias ``your regular lecturer is more qualified for this'' Beck}
    \begin{itemize}
        \item Recalling the definition of an evenly covered set. New notation was introduced, but
            \LaTeX is behind the times.
            Let $E$ and $B$ be topological spaces
            %\hookrightarrow for injectivity
            \begin{align*}
                \phi&:E \twoheadrightarrow B\\
                \forall b \in B,&\ \exists u\ \text{a neighborhood of b}: p^{-1}(u)=\cup_{\alpha}v_\alpha\\
                p_{|v_{\alpha}}&: v_{\alpha}\rightarrow u\\
            \end{align*}
        \item \underline{Fun notation facts:}\\
            $\twoheadrightarrow$ indicates a surjective function\\
            $\hookrightarrow$ indicates an injective function\\
            Combining the two gives you a bijective function, but that symbol doesn't exist
            in latex apparently.
        \item \underline{Example covering:}
            \begin{align*}
                E&=\R\\
                \phi(x)&=(\cos(2\pi x), \sin(2\pi x))\\
                B&=S^{1}\\
            \end{align*}
    \end{itemize}
            \begin{definition} Given a covering map from topological
                spaces $E$ to $B$
                \begin{align*}
                    p: E\rightarrow B\\
                \end{align*}
                    a path in our topological space B,
                \begin{align*}
                    f: I \rightarrow B\\
                \end{align*}
                A \underline{lift} of $f$ is a path, $\tilde{f}: I\rightarrow E$, such that
                $f=p\circ \tilde{f}$
            \end{definition}
    \begin{itemize}
        \item 
            This is theoretically a theorem.\\
            Given covering map $p:E\rightarrow B$, $p(e)=b$,
            $f:I \rightarrow B$ path beginning at b, then there does not exist a left $\tilde{f}$,
            of $f$ beginning at $e$ Read Lemma 54.1 Munkres. (?!?!?)
    \end{itemize}
    \begin{theorem}
        Let the following be so,
        \begin{align*}
            E& \text{ be a topological space}\\
            B& \text{ be a topological space}\\
            p&:E \rightarrow B \text{ a covering map}\\
            f&:I \rightarrow B \text{ path beginning at $b$ }\\
            e &\in E,\ s.t.\ p(e)=b\\
        \end{align*}
        Then there exists a unique path, $\tilde{f}$ in $E$ such that $p\circ \tilde{f}=f$, and
        $\tilde{f}(0)=e$
    \end{theorem}
\end{document}