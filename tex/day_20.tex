\documentclass[../notes.tex]{subfiles}
\begin{document}
\section{Day 20}
\subsection{Free products continued}
\begin{itemize}
    \item \underline{Question:} Let,
        \begin{align*}
            G=\{\dots,a^{-2},a^{-1}a^{0},a^{1}a^{2},\dots \}\\
            H=\{\dots,b^{-2},b^{-1}b^{0},b^{1}b^{2},\dots \}\\
        \end{align*}
        These are both groups under multiplication and they're isomorphic.
        \begin{itemize}
            \item To what ``familiar'' group are $G$ and $H$ isomorphic.
            \item What do elements of $G*H$ look like?
        \end{itemize}
    \item \underline{Answer:}
        \begin{itemize}
            \item 
                \[
                    G\cong H \cong Z
                \]
                (via the isomorphism, $a^i\rightarrow i$ or $b^i \rightarrow i$)
            \item
                Elements of $G*H$ look like,
                \[
                    a^{i_1}b^{j_1}a^{i_2}b^{j_2}\dots a^{i_n}b^{j_n}
                \]
                (Where $i_k, j_k\in \Z$)\\
                This might start with $b^{j_1}$ or it might end with $a^{j_n}$
        \end{itemize}
    \item \underline{Observation:}
\end{itemize}
\subsubsection{Free Products with Amalgamation}
\begin{itemize}
    \item Let $G_1, G_2$ and $H$ be groups, and let,
        \begin{align*}
            \varphi_1:H \rightarrow G_1\\
            \varphi_2:H \rightarrow G_2\\
        \end{align*}
        be homomorphisms\\
        \begin{center}
            \begin{tikzcd}
                G_1 & & G_2\\
                & H \arrow[ul, "\varphi_1"] \arrow[ur, "\varphi_2"] & \\
            \end{tikzcd}
        \end{center}
    \item \underline{Observation:} $\varphi_1$ and $\varphi_2$ induce homomorphisms,
        \begin{align*}
            \tilde{\varphi_1}:H\rightarrow G_1*G_2\\
            \tilde{\varphi_1}= \text{ the word that just contains $\varphi_1(h)$ }\\
        \end{align*}
        and,
        \begin{align*}
            \tilde{\varphi_2}:H\rightarrow G_1*G_2\\
            \tilde{\varphi_2}= \text{ the word that just contains $\varphi_2(h)$ }\\
        \end{align*}
\end{itemize}
\begin{definition}
    The \underline{free product of $G_1$ and $G_2$ amalgamated over $H$} is,
    \begin{align*}
        G_1*_{H}:= \frac{G_1*G_2}{N}\\
    \end{align*}
    where N is the smallest normal subgroup of $G_1*G_2$ containing $\tilde{\varphi_1}(h)\tilde{\varphi_2}(h)^{-1},\ \forall h\in H$
\end{definition}
\begin{itemize}
    \item \underline{Think:} We're setting,
        \begin{align*}
            \tilde{\varphi_1}(h)\tilde{\varphi_2}(h)^{-1}=1\\
            \iff \tilde{\varphi_1}(h)=\tilde{\varphi_2}(h)\\
        \end{align*}
        (the sorts of things in $N$ are $a\tilde{\varphi_1}(h)\tilde{\varphi_2}(h)^{-1}a^{-1}$, e.g.)
    \item \underline{Note:} The notation $G_1*G_2$ doesn't mention $\varpi_1$ and $\varphi_2$
        but it depends on them!
    \item \underline{Ex:}
        \begin{align*}
            G_1=\inpr{a}\\
            G_2=\{1\}\\
            H=\inpr{b}\\
        \end{align*}
        Let's form
        \[
            G_1*_{H}G_2\\
        \]
        where the amalgamation happens over the homomorphisms.
        \begin{align*}
            \varphi_1: \inpr{b}\rightarrow\inpr{a}\\
            \varphi_1(b^i)=a^{2i}\\
        \end{align*}
        and
        \begin{align*}
            \varphi_2: \inpr{b}\rightarrow\{1\}\\
            \varphi_2(b^i)=1\\
        \end{align*}
        Then,
        \begin{align*}
            G_1*_{H}G_2=\frac{G_1*G_2}{
                \text{smallest nomal subgroup containing $\varphi_1(h)\varphi_2^{-1}(h)
                    \forall h\in H$
                }
            }\\
            =\frac{\inpr{a}}{\dots \text{containing $a^2 1\forall i\in \Z$}}=
            \frac{\inpr{a}}{\inpr{a^2}}\\
        \end{align*}
\end{itemize}
\end{document}
