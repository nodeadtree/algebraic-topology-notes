\documentclass[../notes.tex]{subfiles}
\begin{document}
\section{Day 21}
\subsection{Van Kampen Theorem}
\begin{theorem}
    Let $X$ be a topological space such that,
    \[
        X=U\cup V
    \]
    where $U,V$ and $U\cap V$ are all open, path-connected, subsets
    of $X$. Let $x_0\in U\cap V$. Then,
    \[
        \pi_1(X,x_0)\cong\pi_1(U,x_0)*_{\pi_1(U\cap V, x_0)}
        \pi_1(V,x_0)
    \]
    Where amalgamation happens over the homomorphisms
    \[
        i_{U*}:\pi_1(U\cap V, x_0)\rightarrow \pi_1(U, x_0)
    \]
    induced by the inclusion $i_U: U\cap V \rightarrow U$, and
    the homomorphism
    \[
        i_{V*}:\pi_1(U\cap V, x_0)\rightarrow \pi_1(V, x_0)
    \]
    induced by the inclusion $i_V: U\cap V \rightarrow V$.\\
    \begin{center}
        \begin{tikzcd}
            &\pi_1(U,x_0)\arrow[rrd, "j_1"]\arrow[rrrd, bend left=20, "k_1"]& & &\\
            \pi_1(U\cap V, x_0) \arrow[rd, "i_{V*}"] \arrow[ru, "i_{U*}"]& &
            &\pi_1(U,x_0)*_{\pi_1(U\cap V, x_0)}
            \pi_1(V,x_0)\arrow[r, "\phi"]& \pi_1(X,x_0)\\
            &\pi_1(V,x_0)\arrow[rru, "j_2"]\arrow[rrru, bend right=20, "k_2"]& & &\\
        \end{tikzcd}
    \end{center}
\end{theorem}
\begin{itemize}
    \item \underline{Ex:}(Algebra) $\Z*_{\{1\}}\Z$, where the amalgamation happens
        over the homomorphisms,
        \begin{align*}
            \varphi_1: \{1\} \rightarrow \Z,\ \text{(trivial homomorphism)}\\
            \varphi_2: \{1\} \rightarrow \Z,\ \text{(trivial homomorphism)}\\
        \end{align*}
        (The trivial homomorphism just sends everything to the identity)
        By definition,
        \begin{align*}
            \Z*_{\{1\}}\Z &= \frac{\Z * \Z}{
                \underbrace{\text{smallest normal subgroup containing $
                    \tilde{\varphi}_1(h)\tilde{\varphi}_2^{-1}(h)\forall h\in\{1\}$
                }
                }_{\tilde{\varphi}_1(1)\tilde{\varphi}_2^{-1}(1)=a^0(b^0)^{-1}=1}
            }\\
            &=\frac{\inpr{a,b}}{\{1\}}\\
            &=\underbrace{\inpr{a,b}}_{\text{Free group on two generators}}
        \end{align*}
        In general, if $\varphi_1:H\rightarrow G_1$, then, 
        $\tilde{\varphi}_1:H\rightarrow G_1 * G_2$, and 
        \[
            \tilde{\varphi}(h)=\text{the word $\varphi_1(h)$ viewed as a word of length 1}
        \]
    \item \underline{Ex:} (Topology), Note that there's a picture that should be here.
        One of these days I'll go through my notes and add all the missing drawings. Probably
        not gonna do that on the 19th of march though.
        \begin{align*}
            X = \text{Drawing of figure 8}&=\{(x,y)\in \R|(x-1)^2+y^2=1\}\cup
            \{(x,y)\in \R|(x+1)^2+y^2=1\}\\
            U&=\{(x,y)\in X|X\le 1\}\\
            V&=\{(x,y)\in X|X\ge -1\}\\
        \end{align*}
        Let $x_0=(0,0)$. Then, drawings, but mostly the following junk.
        \begin{align*}
            U\simeq S^1\implies \pi_1(U, x_0)\cong \Z\\
            V\simeq S^1\implies \pi_1(V, x_0)\cong \Z\\
            U\cap V\simeq \{(0,0)\}\implies \pi_1(V, x_0)=\{1\}\\
        \end{align*}
        So, the Van Kampen Theorem says,
        \[
            \pi_1(X,x_0)=\Z*_{1}\Z=\inpr{a,p}
        \]
        Pictorially, an element of $\pi_1(X,x_0)$ may look like some as yet undrawn
        picture or,
        \[
            a^3b^5a^{-2}
        \]
        Where $a^1$ corresponds to a clockwise loop around one portion of the figure 8,
        a $a^{-1}$ corresponds to a counterclockwise loop around that same portion, and
        switching to $b^1$ or $b^{-1}$ switches which portion of the figure 8 the loop will be around.
\end{itemize}
\end{document}
