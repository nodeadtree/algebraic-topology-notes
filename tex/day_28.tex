\documentclass[../notes.tex]{subfiles}
\begin{document}
\section{Day 28}
\subsection{$\Delta$-complexes}
\begin{definition}
    Let $X$ be any topological space.
    A \underline{$\Delta$-complex structure} on $X$ is a collection of sets:
    \begin{align*}
        S_0&=\{\sigma_1^0,\sigma_2^0,\dots,\sigma_{k_0}^0\}\\
        S_1&=\{\sigma_1^1,\sigma_2^1,\dots,\sigma_{k_1}^1\}\\
        S_2&=\{\sigma_1^2,\sigma_2^2,\dots,\sigma_{k_2}^2\}\\
        \vdots&\\
        S_n&=\{\sigma_1^n,\sigma_2^n,\dots,\sigma_{k_n}^n\}\\
    \end{align*}
    Where,
    \[
        \underbrace{\sigma_i^n:\Delta^n\rightarrow X}_{
            \text{``$n$-simplex in $X$''}
        }
    \]
    (note that $\Delta^n$ is the standard $n$-simplex)
    and that the maps are continuous such that,
    \begin{enumerate}
        \item ``simplices are glued together only along their boundaries and they glue to form all of $X$''
        \item ``a face of a simplex in $X$ is another simplex''
        \item ``the topology on $X$ is compatible with the topologies on the simplices''
    \end{enumerate}
    \begin{itemize}
        \item \underline{E.g.} In condition 3, we're excluding the possibility of forming
            a set by gluing the simplices then giving that set the trivial topology. CHECK
            IT OUT THER'S A DOODLING.
    \end{itemize}
    \end{definition}
    \begin{itemize}
        \item \underline{Ex:} $X=S^1$
            \begin{align*}
                S_0 &= \{\sigma_1^0,\sigma_2^0\}\\
                \sigma_1^0&:\Delta^0\rightarrow X \text{ constant map at $(1,0)$ }\\
                \sigma_2^0&:\Delta^0\rightarrow X \text{ constant map at $(-1,0)$ }\\
                S_1 &= \{\sigma_1^1,\sigma_2^1\}\\
                \sigma_1^1&:\Delta^1\rightarrow X\\
                \sigma_2^1&:\Delta^1\rightarrow X\\
                \sigma_1^1(t)&=(\cos(\pi t),\sin(\pi t))\\
                \sigma_1^1(t)&=(\cos(\pi t),-\sin(\pi t))\\
            \end{align*}
        \begin{center}
            \begin{tikzpicture}
                \draw (2,2) circle (2cm);
                \draw (4,2) node[right] { $(1,0)$};
                \draw (0,2) node[left] { $(-1,0)$};
                \draw (4,2) node {$\circ$};
                \draw (0,2) node {$\circ$};
                \draw (2,4) node {$<$};
                \draw (2,0) node {$>$};
            \end{tikzpicture}
        \end{center}
    \item \underline{Ex:} $X=S^1$
        \begin{align*}
            S_0 &= \{\sigma_1^0\} \text{ ( constant map at $(1,0)$ ) }\\
            S_1 &= \{\sigma_1^1\}\\
            \sigma_1^1&:\Delta^1 \rightarrow X\\
            \sigma_1^1(t)&=(\cos(2\pi t),\sin(2\pi t))
        \end{align*}
        \begin{center}
            \begin{tikzpicture}
                \draw (2,2) circle (2cm);
                \draw (2,4) node {$<$};
                \draw (4,2) node[right] { $(1,0)$};
                \draw (4,2) node{$\circ$};
            \end{tikzpicture}
        \end{center}
    \item \underline{Ex:} $X=$ torus
        \begin{align*}
            &S_0 = \{\sigma_1^0\}\\
            &\underbrace{S_1 = \{\sigma_1^1,\sigma_2^1,\sigma_3^1\}}_{\text{
                map to $e_1,e_2,e_3$ respectively in the direction shown
                }
            }\\
            &S_2= \{\sigma_1^2,\sigma_2^2\}
        \end{align*}
        Fuckin lost me brah
    \end{itemize}
\end{document}
