\documentclass[../notes.tex]{subfiles}
\begin{document}
\section{Day 25}
\begin{itemize}
    \item Let
        \[
            \sigma = \text{ simplex spanned by }\{(0,1,1),(0,0,4),(1,1,1)\}
        \]
    \item Convince yourself that this is affinely independent
        \begin{itemize}
            \item Draw a picture of $\sigma$\\
        \begin{center}
            \begin{tikzpicture}[line join=bevel,z=2.5, y=2]
                \begin{axis}[
                    axis lines = center,
                    y dir = reverse,
                    xmin=0,xmax=1,
                    ymin=0,ymax=1,
                    zmin=0,zmax=5
                    ]
                    \addplot3+[fill,
                               patch type=polygon,
                               fill opacity=.3,
                               fill color=green]
                    coordinates {
                        (0,1,1)
                        (0,0,4)
                        (1,1,1)
                    } --cycle;
                \end{axis}
            \end{tikzpicture}
        \end{center}
            \item Label the faces of $\sigma$
        \end{itemize}
    \item Note that this is affinely independent because
        \[
            \{(0,-1,3),(1,0,0)\}
        \]
        is linearly independent
    \item The faces of $\sigma$ are,
        \begin{itemize}
            \item The three vertices
            \item The three edges
            \item All of $\sigma$
        \end{itemize}
\end{itemize}
\begin{definition}
    A \underline{simplicial complex} $K$ is a finite set of simplices 
    $\sigma\in \R^N$ such that,
    \begin{itemize}
        \item If $\sigma\in K$, then every face of $\sigma$ is also in $K$
        \item If $\sigma, \tau\in K$ then $\sigma\cap \tau$ is
            a face of both $\sigma$ and $\tau$.
    \end{itemize}
\end{definition}
\newpage
\begin{itemize}
    \item \underline{Ex:}
        \begin{center}
            \begin{tikzpicture}
                \begin{axis}[
                    xmin=-1,xmax=4,
                    ymin=-1,ymax=3
                    ]
                    \addplot+[fill,fill color=blue,draw=black,fill opacity=.3] coordinates 
                    {(0,1) (0,0) (1,0)} --cycle;
                    \addplot+[fill,fill color=orange,draw=black,fill opacity=.3] coordinates 
                    {(0,1) (2,1) (1,0)} --cycle;
                    \addplot+[color=black] coordinates
                    {(2,1) (3,1)} --cycle;
                \end{axis}
            \end{tikzpicture}\\
        \end{center}
    \item \underline{Non-Ex:}\\
        \begin{center}
            \begin{tikzpicture}
                \begin{axis}[
                    xmin=-1,xmax=4,
                    ymin=-1,ymax=3
                    ]
                    \addplot+[fill,fill color=blue,draw=black,fill opacity=.3] coordinates 
                    {(0,1) (0,0) (1,0)} --cycle;
                    \addplot+[fill,fill color=orange,draw=black,fill opacity=.3] coordinates 
                    {(.5,.5) (.5,1.5) (2,1.5)} --cycle;
                \end{axis}
            \end{tikzpicture}\\
        \end{center}
        This is not a simplicial complex, because the intersection of the orange triangle
        and the blue triangle is not a face of both.
        \newpage
    \item \underline{Non-ex:}\\
        \begin{center}
            \begin{tikzpicture}
                \begin{axis}[
                    xmin=-1,xmax=4,
                    ymin=-1,ymax=3
                    ]
                    \addplot+[fill,fill color=blue,draw=black,fill opacity=.3] coordinates 
                    {(0,1) (0,0) (1,0)} --cycle;
                    \addplot+[fill,fill color=orange,draw=black,fill opacity=.3] coordinates 
                    {(.5,.5) (1.5,-.5) (1.5,.5)} --cycle;
                \end{axis}
            \end{tikzpicture}\\
        \end{center}
        Again, this is not a simplicial complex, because the intersection of the orange 
        triangle and the blue triangle is not a face of both.
    \item \underline{Non-ex:}\\
        \begin{center}
            \begin{tikzpicture}
                \begin{axis}[
                    xmin=-1,xmax=4,
                    ymin=-1,ymax=3
                    ]
                    \addplot+[fill,fill color=blue,draw=black,fill opacity=.3] coordinates 
                    {(0,1) (0,0) (1,0)} --cycle;
                    \addplot+[fill,fill color=orange,draw=black,fill opacity=.3] coordinates 
                    {(.15,.15) (.15,.65) (.65,.15)} --cycle;
                \end{axis}
            \end{tikzpicture}\\
        \end{center}
        Yet again, this is not a simplicial complex, because the intersection of the orange 
        triangle and the blue triangle is not a face of both.
    \item
        \underline{Observation:} If $K$ is a simplicial complex, there's an associated
        topological space,
        \[
            |K|=\bigcup_{\sigma\in K}\subseteq \R^N
        \]
        (topological space with the subspace topology from the euclidean topology)\\
        We call $|K|$ the \underline{underlying topological space} of $K$.
\end{itemize}
\subsubsection{Towards simplicial homology}
\begin{definition}
Let,
\begin{align*}
    K=\text{Simplicial complex}\\
    K_n=\{\text{n-simplices in $K$}\subseteq K\\
\end{align*}
The group of \underline{$n$-chains} in $K$ is:
\[
    C_n(K)=\{a_1\sigma_1+a_2\sigma_2+\cdots+a_r\sigma_r| \sigma_i\in K,\ a_i\in \Z\}
\]
(Formal linear combinations, in other words, $C_n(K)\cong \Z^{\#K_n}$)
\end{definition}
\begin{itemize}
    \item \underline{Ex:}
        \begin{align*}
            5\sigma-7\tau\in C_2(K)\\
            -v_1+2v_2-3v_4\in C_0(K)
        \end{align*}
    \item This is a group under addition, e.g.:
        \[
            (5\sigma-7\tau)+(3\sigma+2\tau)=8\sigma-5\tau
        \]
    \item Want to define ``boundary maps'':
        \[
            C_n(K)\rightarrow C_{n-1}(K)
        \]
    \item \underline{Notation:} We denote the simplex spanned by $v_0,\dots,v_n\in\R^n$
        by $[v_0,\dots,v_n]$
\end{itemize}
\begin{definition}
    Let $n\geq 1$. Define the \underline{boundary homomorphism}
    \[
        \partial_n:C_n(K)\rightarrow C_{n-1}(K)
    \]
    on each
    \[
        \sigma = [v_0,v_1,\dots,v_n]
    \]
    by:
    \[
        \partial_n(\sigma)=\sum_{j=0}^{n}(-1)^j[v_0,\dots,\hat{v}_j,\dots,v_n]
    \]
    With $\hat{v}_j$ indicating that we remove this from the simplex
\end{definition}
\newpage
\begin{itemize}
    \item \underline{Ex:}
        \filled{(0,1) (0,0) (1,0)}
        \begin{align*}
            \partial_2&:C_2(K)\rightarrow C_1(K)\\
            \partial([(0,1),(0,0),(1,0)])&=[(0,0),(1,0)]-[(0,1),(1,0)]+[(0,1),(0,0)]
        \end{align*}
\end{itemize}
\end{document}
